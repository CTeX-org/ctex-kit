\PassOptionsToPackage{scheme=plain, linespread=1.1}{ctex}
\documentclass{ctxdoc}
\setCJKmainfont[Language=Chinese Simplified]{Source Han Serif}

\ctexset{section/name={}}
\pagestyle{headings}

\makeatletter
\RecustomVerbatimEnvironment{frameverb}{Verbatim}{%
  gobble=2,
  frame=single, framesep=8pt,
  listparameters=
    \setlength\topsep{\medskipamount}%
    \appto\FV@EndList{\nointerlineskip}}
\RecustomVerbatimEnvironment{ctexexam}{Verbatim}{%
  gobble=2,
  frame=single, framesep=10pt,
  label=\rule{0pt}{12pt}\textnormal{\bfseries Example \arabic{ctexexam}},
  listparameters=%
    \setlength\topsep{\bigskipamount}%
    \refstepcounter{ctexexam}\ctexexamlabelref
    \appto\FV@EndList{\nointerlineskip}}
\newlist{psopt}{description}{3}
\setlist[psopt]{%
  font=\mdseries\ttfamily, align=right,
  labelsep=.5em, leftmargin=4.5em, labelindent=0pt}
\newcommand\PSKeyVal[2]{%
  \item[#1]\makebox[4em][l]{\meta{#2}}\ignorespaces}
\AtBeginDocument{\DeleteShortVerb{\"}}
\newcommand\USV[1]{\texttt{U+#1}}
\makeatother

\title{\bfseries The \pkg{xeCJK} package}
\author{\href{http://www.ctex.org}{CTEX.ORG}}
\date{2019/04/07\qquad3.7.2\thanks{\ctexkitrev{\ExplFileVersion}.}}

\begin{document}

\maketitle
\tableofcontents

\section{Introduction}

\pkg{xeCJK} is a \XeLaTeX{} package for typesetting Chinese (C), Japanese (J)
and Korean (K) scripts. Its main functions are:
\begin{enumerate}
  \item Set fonts for CJK scripts and other scripts separately.
  \item Automatically ignore the spaces between CJK characters while keep other
    spaces; allow ling-breaking between non-punctuation CJK and Latin
    characters (a--z, A--Z).
  \item Provide multiple punctuation styles: fullwidth, halfwidth, kaiming,
    EOL-halfwidth and CCT style.
  \item Automatically adjust the spacing between CJK and other characters.
\end{enumerate}

\pkg{xeCJK} uses some newest features of \XeTeX{}, hence requires \XeTeX{}
after version 0.9995.0 (2009/06/29) or higher. \pkg{xeCJK} relies on
\package{l3kernel} and \package{l3packages} in \LaTeXiii{} project. In addition,
\pkg{xeCJK} also needs package \package{fontspec} to use system fonts.
\pkg{xeCJK} will load the packages above wehn needed.

The original author of \pkg{xeCJK} is Sun Wenchang (孙文昌). From May 2009,
this package was merged into \ctexkit{} project. Currently, the primary
maintainers are Liu Haiyang\footnote{\email{leoliu.pku@gmail.com}} (刘海洋) and
Li Qing\footnote{\email{sobenlee@gmail.com}} (李清).

\section{Basic Usage}

As other \LaTeX{} packages, using
\begin{frameverb}
  \usepackage{xeCJK}
\end{frameverb}
in the preamble will load the \pkg{xeCJK} package. After this declaration,
CJK scripts can be then used as long as proper fonts are selected.

\pkg{xeCJK} may be used in all kinds of document classes. A minimal example is:
\begin{ctexexam}
  \documentclass{article}
  \usepackage{xeCJK}
  \setCJKmainfont{SimSun}

  \begin{document}
  中文 \LaTeX 示例。
  \end{document}
\end{ctexexam}
In the example above, SimSun (the default Simplified Chinese font in Microsoft
Windows) is used for Chinese font. To build this example, this font must have
been installed in your system, the file itself saved in UTF-8 encoding, and
\XeLaTeX{} used for compilation.

\pkg{xeCJK} only offers fundamental CJK language support such as font selection
and punctuation handling. For Chinese document, you can use the \package{ctex}
package or classes for higher-level support. It will load \pkg{xeCJK}
automatically (when using \XeLaTeX{}), set Chinese fonts, as well as offering
further localization support. You can find more details in the documentation of
\package{ctex} bundle.

\pkg{xeCJK} provides a wide variety of options, which can either be declared as
package options or specified with \tn{xeCJKsetup}, as described in
Section~\ref{subsec:opts}. Besides \tn{setCJKmainfont} command, \pkg{xeCJK}
also defines many other commands to set and choose CJK fonts, which can be
found in Section~\ref{subsec:fontset}. Other functionalities will also be
described below in more details. In the |example| folder under the same
directory of this documentation, there are also some examples for references.

\section{User's Manual}

\subsection{Package options}
\label{subsec:opts}

The package options of \pkg{xeCJK} have the form of \meta{key}|=|\meta{val}.
You can specify the options when loading the package, or use \tn{xeCJKsetup}
command afterwards. \pkg{xeCJK} loads \pkg{fontspec} package internally. So you
can use its options when loading \pkg{xeCJK}, and \pkg{xeCJK} will pass them
to \pkg{fontspec}.

\begin{function}{\xeCJKsetup}
  \begin{syntax}
    \tn{xeCJKsetup} \{\meta{key_1}=\meta{val_1}, \meta{key_2}=\meta{val_2}, ...\}
  \end{syntax}
  \meta{key_1}, \meta{key_2}, \emph{etc.} are options, while \meta{val_1},
  \meta{val_2}, \emph{etc.} are the corresponding values. For example,
  \begin{ctexexam}
  \usepackage[PunctStyle=kaiming]{xeCJK}
  \end{ctexexam}
  is equivalent to
  \begin{ctexexam}
  \usepackage{xeCJK}
  ......
  \xeCJKsetup{PunctStyle=kaiming}
  \end{ctexexam}
\end{function}

The \expstar{} after some options or commands means that this option or command
can only be used in the preamble. Symbol \rexpstar{} indicates that the option
or command can only used in the preamble as well, and only affect the CJK font
defined afterwards. Other options or commands without any special marks can be
used in both the preamble or the document body. Boldface represents the default
settings of \pkg{xeCJK}.

\begin{function}[EXP,added=2012-11-22]{LocalConfig}
  \begin{syntax}
    LocalConfig = \Arg{\TTF|name}
  \end{syntax}
  Whether or not use the local configuration file
  \texttt{xeCJK-\meta{name}.cfg}. \meta{name} can be any string that contains
  no spaces and should make the filename valid. If set to be |true|, then
  |xeCJK.cfg| is used; if set to be |false|, then no configuration file
  will be loaded. You can save some settings of \pkg{xeCJK} mentioned below
  (e.g., setting common used CJK fonts, modifying range of characters and
  defining new punctuation styles) into file \texttt{xeCJK-\meta{name}.cfg},
  then put this file to a proper location of |TDS| directory. \TeXLive{} users
  may create the following directory and then put \texttt{xeCJK-\meta{name}.cfg}
  inside:
  \begin{frameverb}
  texlive/texmf-local/tex/xelatex/xecjk
  \end{frameverb}
  After that, execute |mktexlsr| in the command line, so that the \TeX{}
  system may find it.
\end{function}

Note that only the |LocalConfig| option above must be set when loading
\pkg{xeCJK} package. It cannot be used with command \tn{xeCJKsetup}.

\begin{function}{xeCJKactive}
  \begin{syntax}
    xeCJKactive = \meta{\TTF}
  \end{syntax}
  Turn on/off the special treatment for CJK scripts. In fact, this option will
  turn on/off the entire character classes system of \XeTeX{}, and all packages
  relying on it will be affected.
\end{function}

\begin{function}{CJKspace}
  \begin{syntax}
    CJKspace = \meta{\TFF}
  \end{syntax}
  By default, \pkg{xeCJK} will ignore spaces between CJK characters. This option
  will keep the spaces between them (for example, when typesetting Korean).
\end{function}

\begin{function}[EXP,updated=2016-05-04]{CJKmath}
  \begin{syntax}
    CJKmath = \meta{\TFF}
  \end{syntax}
  Whether or not allow CJK characters in math mode. When this option is used,
  CJK characters can be typed in math mode directly. The \pkg{url} package puts
  URLs in a special math environment, so this option need to be specified when
  you want to use CJK characters correctly inside the parameter of \tn{path} or
  other commands.
\end{function}

\begin{function}{CJKglue}
  \begin{syntax}
    CJKglue = \{\tn{hskip} 0pt plus 0.08\tn{baselineskip}\}
  \end{syntax}
  Set the |glue| between CJK characters. The above is the default value of
  \pkg{xeCJK}. Normally, you should simply keep the default value and not
  modify it unless there are special needs (such as changing the distance
  between characters). If you want to modify this value, the |glue| should be
  flexible for line justification.
\end{function}

\begin{function}{CJKecglue}
  \begin{syntax}
    CJKecglue = \Arg{glue}
  \end{syntax}
  Set the glue between CJK characters and other characters or inline math
  formulas. The default value is a space. If you want to modify this value,
  the |glue| should be flexible as well. Note that the \meta{glue} set here
  only affects spaces add automatically by \pkg{xeCJK}. Explicit spaces between
  CJK and other characters in source code will not be affected (output
  directly). In some cases, \pkg{xeCJK} is unable to adjust the spaces
  correctly, and manual inserting spaces is necessary.
\end{function}

\begin{function}{xCJKecglue}
  \begin{syntax}
    xCJKecglue = \Arg{\TFF|glue}
  \end{syntax}
  By default, \pkg{xeCJK} does not adjust the spaces directly inserted between
  CJK characters and other characters. Use this option if you want to adjust
  it. When set to be |true|, spaces between CJK and other characters will be
  replaced by |CJKecglue|; when set to be \meta{glue}, |CJKecglue| will be
  overridden and all the spaces between CJK and other characters will be set as
  \meta{glue}.
\end{function}

\begin{function}[updated=2013-06-26]{CheckSingle}
  \begin{syntax}
    CheckSingle = \meta{\TFF}
  \end{syntax}
  Whether or not avoid a single CJK character occupying the last line of a
  paragraph. This option can work properly only when the last character
  of a paragraph is a CJK character or punctuation, and the two characters
  before it are both ordinary characters. When any of the three is a parameter
  of a control sequence, then in general it may not be handled correctly.
\end{function}

\begin{function}[added=2015-04-08]{WidowPenalty}
  \begin{syntax}
    WidowPenalty = \Arg{penalty|(10000)}
  \end{syntax}
  Set the penalty between the last three CJK characters after using
  |CheckSingle|. The default value is \num{10000}, \emph{i.e.}, line
  break between them is forbidden.
\end{function}

\begin{function}[added=2012-12-06]{PlainEquation}
  \begin{syntax}
    PlainEquation = \meta{\TFF}
  \end{syntax}
  If you use |$$...$$| for display equations, then this option should be used,
  so that |CheckSingle| option can work properly. We recommend to use |\[...\]|
  for display equations.
\end{function}

\begin{function}[added=2012-12-04]{NewLineCS,NewLineCS+,NewLineCS-}
  \begin{syntax}
    NewLineCS = \{ \tn{par} \tn{[} \}
  \end{syntax}
  Set the control sequences that may result in line break, so that 
  |CheckSingle| option can work properly. The list above is the default value
  of \pkg{xeCJK}.
\end{function}

\begin{function}[added=2012-12-04]{EnvCS,EnvCS+,EnvCS-}
  \begin{syntax}
    EnvCS = \{ \tn{begin} \tn{end} \}
  \end{syntax}
  Set the control sequence that can begin or end a \LaTeX{} environment, so
  that |CheckSingle| option can work properly. The list above is the default
  value of \pkg{xeCJK}.
\end{function}

\begin{function}[updated=2012-12-06]{InlineEnv,InlineEnv+,InlineEnv-}
  \begin{syntax}
    InlineEnv = \{\meta{env_1}, \meta{env_2}, \meta{env_3}, ...\}
  \end{syntax}
  When the |CheckSingle| option is used, \pkg{xeCJK} will treat the beginning
  (|\begin{...}|) and ending (|\end{...}|) of \LaTeX{} environments after a CJK
  character as line breaks. When there are some special \LaTeX{} environments
  that do not result in line breaks, they should be declared in this option,
  so that |CheckSingle| can work properly.
\end{function}

\begin{function}{AutoFallBack}
  \begin{syntax}
    AutoFallBack = \meta{\TFF}
  \end{syntax}
  When there are several rarely-used characters in the document that the
  default font family does not contain, this option can be used to output these
  characters with fallback font. The settings of fallback fonts will be
  discussed in Section~\ref{subsec:fontset}.
\end{function}

\begin{function}[rEXP]{AutoFakeBold}
  \begin{syntax}
    AutoFakeBold = \Arg{\TFF|number}
  \end{syntax}
  Globally set whether or not to use
  \textbf{\addfontfeatures{AutoFakeBold}fake bold} when the real bold were not
  declared for the fonts. When set to be \meta{number}, fake bold will be used
  and the number stands for the default embolden factor.
\end{function}

\begin{function}[rEXP]{AutoFakeSlant}
  \begin{syntax}
    AutoFakeSlant = \Arg{\TFF|number}
  \end{syntax}
  Globally set whether or not to use
  \textit{\addfontfeatures{AutoFakeSlant}fake slant} when the real italic or
  slant were not declared for the fonts. When set to be \meta{number}, fake
  slant will be used and the number stands for the default slant factor. The
  range of the slant factor is $[-0.999, 0.999]$.
\end{function}

\begin{function}[rEXP]{EmboldenFactor}
  \begin{syntax}
    EmboldenFactor = \Arg{number|(4)}
  \end{syntax}
  Set the default embolden factor for fake bold.
\end{function}

\begin{function}[rEXP]{SlantFactor}
  \begin{syntax}
    SlantFactor = \Arg{number|(0.167)}
  \end{syntax}
  Set the default slant factor for fake slant. The range is $[-0.999, 0.999]$.
\end{function}

\begin{function}[updated=2012-11-10]{PunctStyle}
  \begin{syntax}
    PunctStyle = \Arg{(quanjiao)|banjiao|kaiming|hangmobanjiao|CCT|plain|...}
  \end{syntax}
  Set the punctuation style. The pre-defined styles in \pkg{xeCJK} are:
  \begin{optdesc}
    \item[quanjiao] Fullwidth style (全角式): all punctuation marks have the
      same width as 1 CJK character (汉字), while two adjacent punctuation
      marks take the width of 1.5 CJK characters.
    \item[banjiao] Halfwidth style (半角式): all punctuation marks have the
      same width as 0.5 CJK character.
    \item[kaiming] Kaiming style (开明式): punctuation marks at end-sentence
      use fullwidth style, while others use halfwidth style.
    \item[hangmobanjiao] EOL-halfwidth (行末半角式): all punctuation marks have
      the same width as 1 CJK character, except those at the end of lines.
      Their width will be the same as 0.5 CJK character for purpose of visual
      justification.
    \item[CCT] CCT style: all punctuation marks take a slightly smaller width
      of 1 CJK character.
    \item[plain] The width of punctuation marks will not adjusted.
  \end{optdesc}
  You can use the \tn{xeCJKDeclarePunctStyle} command described in
  Section~\ref{subsec:punctstyle} to define new punctuation styles.
\end{function}

\begin{function}[added=2018-01-24]{PunctFamily}
  \begin{syntax}
    PunctFamily = \Arg{(false)|family}
  \end{syntax}
  By default, the font of CJK punctuation are the same as CJK text. The option
  \opt{PunctFamily} is used to set font for CJK punctuation separately.
  \meta{family} should be declared by \tn{setCJKfamilyfont} or
  \tn{newCJKfontfamily}, as described later. \opt{false} means to disable this
  setting and let CJK punctuation and text use the same font.
\end{function}

\begin{function}[EXP]{KaiMingPunct,KaiMingPunct+,KaiMingPunct-}
  \begin{syntax}
    KaiMingPunct = \Arg{\normalfont\CJKfamily+{rm}\ ^^^^3002 ^^^^ff0e ^^^^ff1f ^^^^ff01 }
  \end{syntax}
  Set the end-sentence punctuation marks used in |kaiming| style. The |+| and
  |-| symbol after |KaiMingPunct| means to add or delete punctuation marks from
  the existing list. The |LongPunct| and |MiddlePunct| options below have the
  similar syntax.

  By default, |KaiMingPunct| is set to be the following characters:
  \begin{center}
    \begin{tabular}{ccc}
      \toprule
        Unicode code point & Character & Name \\
      \midrule
        \USV{3002} & {\CJKfamily+{rm}^^^^3002} & Ideographic full stop      \\
        \USV{FF0E} & {\CJKfamily+{rm}^^^^ff0e} & Fullwidth full stop        \\
        \USV{FF1F} & {\CJKfamily+{rm}^^^^ff1f} & Fullwidth question mark    \\
        \USV{FF01} & {\CJKfamily+{rm}^^^^ff01} & Fullwidth exclamation mark \\
      \bottomrule
    \end{tabular}
  \end{center}
\end{function}

\begin{function}[EXP]{LongPunct,LongPunct+,LongPunct-}
  \begin{syntax}
    LongPunct = \Arg{\normalfont\CJKfamily+{rm}\ ^^^^2014 ^^^^2e3a ^^^^2025 ^^^^2026 }
  \end{syntax}
  Set the long punctuation marks, such as em dash ``——'' or ellipsis ``……''.
  Line break is allowed before or after them, but not between them.

  By default, |LongPunct| is set to be the following characters:
  \begin{center}
    \begin{tabular}{ccc}
      \toprule
        Unicode code point & Character & Name \\
      \midrule
        \USV{2014} & {\CJKfamily+{rm}^^^^2014} & Em dash             \\
        \USV{2E3A} & {\CJKfamily+{rm}^^^^2e3a} & Two-em dash         \\
        \USV{2025} & {\CJKfamily+{rm}^^^^2025} & Two dot leader      \\
        \USV{2026} & {\CJKfamily+{rm}^^^^2026} & Horizontal ellipsis \\
      \bottomrule
    \end{tabular}
  \end{center}
\end{function}

\begin{function}[EXP]{MiddlePunct,MiddlePunct+,MiddlePunct-}
  \begin{syntax}
    MiddlePunct = \Arg{\normalfont\CJKfamily+{rm}\ ^^^^2013 ^^^^2014 ^^^^2e3a ^^^^2027 ^^^^00b7 ^^^^30fb ^^^^ff65 }
  \end{syntax}
  Set the middle punctuation marks, such as the middle dot ``·''. For middle
  punctuation marks between CJK characters, \pkg{xeCJK} will adjust the space
  before and after them to make sure that they appear centered, according to
  the punctuation style. When a middle punctuation mark appears at the end of
  line, line break is allowed after it, but not before it.

  By default, |MiddlePunct| is set to be the following characters:
  \begin{center}
    \begin{tabular}{ccc}
      \toprule
        Unicode code point & Character & Name \\
      \midrule
        \USV{2013} & {\CJKfamily+{rm}^^^^2013} & En dash                       \\
        \USV{2014} & {\CJKfamily+{rm}^^^^2014} & Em dash                       \\
        \USV{2E3A} & {\CJKfamily+{rm}^^^^2e3a} & Two-em dash                   \\
        \USV{2027} & {\CJKfamily+{rm}^^^^2027} & Hyphenation point             \\
        \USV{00B7} & {\CJKfamily+{rm}^^^^00b7} & Middle dot                    \\
        \USV{30FB} & {\CJKfamily+{rm}^^^^30fb} & Katakana middle dot           \\
        \USV{FF65} & {\CJKfamily+{rm}^^^^ff65} & Halfwidth katakana middle dot \\
      \bottomrule
    \end{tabular}
  \end{center}
\end{function}

\begin{function}[EXP]{PunctWidth}
  \begin{syntax}
    PunctWidth = \Arg{length}
  \end{syntax}
  By default, \pkg{xeCJK} will calculate the width of punctuation marks based
  on punctuation style. If the default settings are not satisfying, you may use
  this option. In order that the width of punctuation marks can change
  according to font size, it is preferred to use a relative unit like |em| for
  the \meta{length}, rather than an absolute unit like |pt|. The settings here
  can be used for all punctuation styles except |plain|. In addition, the
  settings apply to all CJK punctuation marks. If you wang to set the width of
  a specific range of punctuation marks, use \tn{xeCJKsetwidth} command
  described in Section~\ref{subsec:punct}.
\end{function}

\begin{function}[EXP,added=2013-08-22]{PunctBoundWidth}
  \begin{syntax}
    PunctBoundWidth = \Arg{length}
  \end{syntax}
  Similar to |PunctWidth|, but only set the width of punctuation at the start
  or end of a line.
\end{function}

\begin{function}{AllowBreakBetweenPuncts}
  \begin{syntax}
    AllowBreakBetweenPuncts = \meta{\TFF}
  \end{syntax}
  By default, \pkg{xeCJK} forbids line break between adjacent CJK right/left
  punctuation marks. This option can be used to change this setting.
\end{function}

\begin{function}[updated=2016-05-13]{RubberPunctSkip}
  \begin{syntax}
    RubberPunctSkip = \meta{\TTF|plus|minus}
  \end{syntax}
  By default, the skip before/after a punctuation mark is flexible. It can
  stretch to the width of original side bearing, or shrink to the side bearing
  on the other side. When the option is set to be |plus|, only stretching is
  allowed; when set to be |minus|, only shrinking is allowed; when set to be
  |false|, this feature is disabled so that the space before/after a
  punctuation mark is fixed.
\end{function}

\begin{function}[added=2012-12-02]{CheckFullRight}
  \begin{syntax}
    CheckFullRight = \meta{\TFF}
  \end{syntax}
  Some control sequences do not allow line break before them. However, line
  break is always enabled after a single CJK right punctuation mark by default.
  As a result, when these control sequence appears after such punctuation
  marks, an unexpected line break may occur. The situation can be avoided with
  this option.
\end{function}

\begin{function}[added=2012-12-02]{NoBreakCS,NoBreakCS+,NoBreakCS-}
  \begin{syntax}
    NoBreakCS = \{ \tn{footnote} \tn{footnotemark} \tn{nobreak} \}
  \end{syntax}
  Set the control sequences that do not allow line break after a CJK right
  punctuation. The default value of \pkg{xeCJK} is listed above. If such
  control sequences only appear several times in the document, it may not be
  necessary to set this option. Instead, you can use \tn{xeCJKnobreak}
  described in Section~\ref{subsec:others} to control them manually.
\end{function}

\begin{function}[updated=2013-11-16]{Verb}
  \begin{syntax}
    Verb = \meta{\TF|(env)|env+}
  \end{syntax}
  Set the space between CJK and other characters in verbatim environments.
  \begin{optdesc}
    \item[true] Do not adjust the space between CJK and other characters in
      \tn{verb} command or |verbatim| environment.
    \item[env] Calculate the space within CJK characters, or between CJK and
      other characters automatically, in order to keep the code aligned.
    \item[env+] Similar as |env|, but add the settings of normal text into
      \tn{verb} command.
    \item[false] Do not adjust the space.
  \end{optdesc}
  The above values, except |false|, all forbid automatic line break between
  CJK and other characters. This option is actually valid for all commands
  using \tn{verbatim@font} command. More general cases can be handled with
  \tn{xeCJKVerbAddon} described in Section~\ref{subsec:others}.
\end{function}

\begin{function}[rEXP,added=2014-03-01]{LoadFandol}
  \begin{syntax}
    LoadFandol = \meta{\TTF}
  \end{syntax}
  Whether or not to use Fandol font when CJK fonts are not declared in the
  preamble. To use this option, the \package{Fandol} font series should be
  installed.
\end{function}


\subsection{Font settings and selection}
\label{subsec:fontset}

\begin{function}[EXP,updated=2016-11-18]{\setCJKmainfont}
  \begin{syntax}
    \tn{setCJKmainfont} \Arg{font name}\oarg{font features} or\\
    \tn{setCJKmainfont} \oarg{font features} \Arg{font name}
  \end{syntax}
  Set the CJK font corresponding to roman family, affecting
  \tn{rmfamily} and \tn{textrm}. The parameters are inherited from
  package \pkg{fontspec}; \meta{font features} selects property options
  of the font, and \meta{font name} can be either a font's family name
  or its filename. For searching of font names, see
  Section~\ref{subsubsec:fontsearch}; for other font features not
  mentioned here, see the documentation of \pkg{fontspec}. However,
  \pkg{xeCJK} modified |AutoFakeBold| and |AutoFakeSlant| options to
  work with global settings mentioned above.

  For compatibility, the parameter for font features can be put either
  before or after font name. If it is put after, there should be no
  line break between.
\end{function}

\begin{function}[label = ]{AutoFakeBold,AutoFakeSlant}
  \begin{syntax}
    AutoFakeBold  = \Arg{\TF|number}
    AutoFakeSlant = \Arg{\TF|number}
  \end{syntax}
  Locally set the fake bold and fake slant property of the current
  font family. If it is absent, the global setting is used.
\end{function}

\begin{function}[added=2013-06-07]{Mapping}
  \begin{syntax}
    Mapping = \Arg{fullwidth-stop|full-stop|han-trad|han-simp|...}
  \end{syntax}
  \pkg{xeCJK} offers the four
  \href{http://scripts.sil.org/teckit}{TECKit} mapping files above,
  which can be used through \texttt{Mapping} option at font
  declaration. \texttt{fullwidth-stop} maps a normal CJK full
  stop ``。'' to a fullwidth solid full stop ``.'', and \texttt{full-stop}
  has an opposite effect. \texttt{han-trad} maps simplified Chinese
  characters to traditional ones, and \texttt{han-simp} has an
  opposite effect. It should be noticed that the conversions are only
  based on individual characters, and might not be accurate. For
  example, ``发挥'' and ``头发'' in simplified Chinese becomes ``發揮'' and
  ``頭發'' respectively, while the latter should be ``頭髮''. Users may also
  make new mapping files based on actual needs. See the documentation
  of TECKit for more information.
\end{function}

\begin{function}[EXP,updated=2016-11-18]{\setCJKsansfont}
  \begin{syntax}
    \tn{setCJKsansfont} \Arg{font name}\oarg{font features} or\\
    \tn{setCJKsansfont} \oarg{font features} \Arg{font name}
  \end{syntax}
  Set the CJK font corresponding to sans-serif family, affecting
  \tn{rmfamily} and \tn{textrm}.
\end{function}

\begin{function}[EXP,updated=2016-11-18]{\setCJKmonofont}
  \begin{syntax}
    \tn{setCJKmonofont} \Arg{font name}\oarg{font features} or\\
    \tn{setCJKmonofont} \oarg{font features} \Arg{font name}
  \end{syntax}
  Set the CJK font corresponding to typewriter family, affecting
  \tn{rmfamily} and \tn{textrm}.
\end{function}

\begin{function}[EXP,updated=2016-11-18]{\setCJKfamilyfont}
  \begin{syntax}
    \tn{setCJKfamilyfont} \Arg{family} \Arg{font name}\oarg{font features} or\\
    \tn{setCJKfamilyfont} \Arg{family} \oarg{font features} \Arg{font name}
  \end{syntax}
  Declare new CJK family \meta{family} with appropriate fonts.
\end{function}

\begin{function}[updated=2012-10-27]{\CJKfamily}
  \begin{syntax}
    \tn{CJKfamily}  \Arg{family}
    \tn{CJKfamily} + \Arg{family}
    \tn{CJKfamily} - \Arg{family}
  \end{syntax}
  Switch CJK font families in the document. \meta{family} must be
  declared before using. \tn{CJKfamily} is only effective on CJK
  characters, \tn{CJKfamily}|+| is effective on all characters,
  and \tn{CJKfamily}|-| is effective on non-CJK characters. When the
  parameter of \tn{CJKfamily}|+| or \tn{CJKfamily}|-| is empty, the
  the current CJK font family is used.
\end{function}

\begin{function}[EXP,updated=2016-11-18]{\newCJKfontfamily}
  \begin{syntax}
    \tn{newCJKfontfamily} \oarg{family} \cs{\meta{font-switch}} \Arg{font name}\oarg{font features} or\\
    \tn{newCJKfontfamily} \oarg{family} \cs{\meta{font-switch}} \oarg{font features} \Arg{font name}
  \end{syntax}
  Declare new CJK family CJK \meta{family} with appropriate fonts,
  as well as defining a command \cs{\meta{font-switch}} to switch to the CJK
  font family. \meta{family} is optional; when it is omitted,
  \meta{family} would be the same as \meta{font-switch}.
\end{function}

  In fact, \tn{newCJKfontfamily} is the combination of
  \tn{setCJKfamilyfont} and \tn{CJKfamily}. For example,
  \begin{ctexexam}
  \newCJKfontfamily[song]\songti{SimSun}
  \end{ctexexam}
  is equivalent to
  \begin{ctexexam}
  \setCJKfamilyfont{song}{SimSun}
  \newcommand*{\songti}{\CJKfamily{song}}
  \end{ctexexam}

\begin{function}[updated=2016-11-18]{\CJKfontspec}
  \begin{syntax}
    \tn{CJKfontspec} \Arg{font name}\oarg{font features} or\\
    \tn{CJKfontspec} \oarg{font features} \Arg{font name}
  \end{syntax}
  Declare new CJK font family in the document, and use it immediately.
\end{function}


\begin{function}[rEXP]{\defaultCJKfontfeatures}
  \begin{syntax}
    \tn{defaultCJKfontfeatures} \Arg{font features}
  \end{syntax}
  Set global options for all CJK font families. For example, using
  \begin{ctexexam}
  \defaultCJKfontfeatures{Scale=0.962216}
  \end{ctexexam}
  will shrink all CJK font to |0.962216| times of the original size.
  The initial setting of \pkg{xeCJK} is
  \begin{frameverb}
  \defaultCJKfontfeatures{Script=CJK}
  \end{frameverb}
\end{function}

\begin{function}[updated=2013-06-30]{\addCJKfontfeatures}
  \begin{syntax}
    \tn{addCJKfontfeatures}   \Arg{font features}
    \tn{addCJKfontfeatures} * \Arg{font features}
    \tn{addCJKfontfeatures}   \oarg{block_1, block_2, ...} \Arg{font features}
    \tn{addCJKfontfeatures} * \oarg{block_1, block_2, ...} \Arg{font features}
  \end{syntax}
  Temporarily add features to current CJK font faily. The first command
  is only effective on the current CJK main block; the second one is
  effective on both main block and all other blocks; the third one is
  effective on the appointed blocks; The fourth one is effective on
  both the main block and the appointed blocks. For example, using
  \begin{ctexexam}
  \addCJKfontfeatures{Scale=1.1}
  \end{ctexexam}
  enlarges the main block of current CJK font to |1.1| times.
\end{function}

\begin{function}{\CJKrmdefault}
  CJK font family used in \tn{textrm} and \tn{rmfamily}. The default
  is |rm|.
\end{function}

\begin{function}{\CJKsfdefault}
  CJK font family used in \tn{textsf} and \tn{sffamily}. The default
  is |sf|.
\end{function}

\begin{function}{\CJKttdefault}
  CJK font family used in \tn{texttt} and \tn{ttfamily}. The default
  is |tt|.
\end{function}

\begin{function}[updated=2013-01-01]{\CJKfamilydefault}
  CJK font family used in \tn{textnormal} and \tn{normalfont},
  analogous to \tn{familydefault}. The initial value is
  \tn{CJKrmdefault}. If it is not modified in the preamble,
  \pkg{xeCJK} will update \tn{CJKfamilydefault} automatically according
  to Western fonts. Therefore, using
  \begin{frameverb}
  \renewcommand\familydefault{\sfdefault}
  \end{frameverb}
  in the preamble modifies both the CJK font and Western font to
  sans-serif family.
\end{function}

\begin{function}[EXP,updated=2016-11-18]{\setCJKmathfont}
  \begin{syntax}
    \tn{setCJKmathfont} \Arg{font name}\oarg{font features} or\\
    \tn{setCJKmathfont} \oarg{font features} \Arg{font name}
  \end{syntax}
  Set CJK family in math mode. If |CJKmath| is used, but this command
  is not used, then \tn{CJKfamilydefault} will be used in math mode
  instead.
\end{function}

\begin{function}[EXP, label=, updated=2016-11-18]{\setCJKfallbackfamilyfont}
  \begin{syntax}
    \tn{setCJKfallbackfamilyfont} \Arg{family} \Arg{font name}\oarg{font features} or\\
    \tn{setCJKfallbackfamilyfont} \Arg{family} \oarg{font features} \Arg{font name}
  \end{syntax}
  Set the fallback font of the CJK font family \meta{family}. For
  example,
  \begin{ctexexam}
  \setCJKmainfont{SimSun}
  \setCJKfallbackfamilyfont{\CJKrmdefault}{SimSun-ExtB}
  \end{ctexexam}
  sets |SimSun-ExtB| as the fall back font of |SimSun|.
\end{function}

\begin{function}{FallBack}
  \begin{syntax}
    FallBack = \{\oarg{font features}\Arg{font name}\}
  \end{syntax}
  \pkg{xeCJK} adds |FallBack| to \meta{font features}, so that fallback
  font may be declared the same time as the main font. The example
  above is thus equivalent to:
  \begin{ctexexam}
  \setCJKmainfont[FallBack=SimSun-ExtB]{SimSun}
  \end{ctexexam}
  When the value of |FallBack| is empty, the command itself sets the
  fallback font. For example,
  \begin{ctexexam}
  \setCJKmainfont[FallBack,AutoFakeBold,Scale=.97]{SimSun-ExtB}
  \end{ctexexam}
  is equivalent to
  \begin{ctexexam}
  \setCJKfallbackfamilyfont{\CJKrmdefault}[AutoFakeBold,Scale=.97]{SimSun-ExtB}
  \end{ctexexam}
\end{function}

\begin{function}[EXP,updated=2013-06-30]{\setCJKfallbackfamilyfont}
  \begin{syntax}
    \tn{setCJKfallbackfamilyfont} \Arg{family}
       \  \{
       \    \{\oarg{font features_1} \Arg{font name_1}\} ,
       \    \{\oarg{font features_2} \Arg{font name_2}\} ,
       \     ......
       \  \}\oarg{common font features} or\\
    \tn{setCJKfallbackfamilyfont} \Arg{family} \oarg{common font features}
       \  \{
       \    \{\oarg{font features_1} \Arg{font name_1}\} ,
       \    \{\oarg{font features_2} \Arg{font name_2}\} ,
       \     ......
       \  \}
  \end{syntax}
  \tn{setCJKfallbackfamilyfont} can also be used to set multi-level
  fallbacks. For example, the declaration
  \begin{ctexexam}
  \setCJKmainfont[AutoFakeBold,AutoFakeSlant]{KaiTi_GB2312}
  \setCJKfallbackfamilyfont{\CJKrmdefault}[AutoFakeSlant]
    { [BoldFont=SimHei]{SimSun} ,
      [AutoFakeBold]   {SimSun-ExtB} }
  \end{ctexexam}
  means that |SimSun| is the fallback font of |KaiTi_GB2312|, and
  |SimSun-ExtB| is the fall back font of |SimSun|. When a character is
  missing in the current font and all its fallback fonts, then the
  fallback font of \tn{CJKfamilydefault} will be used.
\end{function}

\subsubsection{Font searching under \XeTeX{}}
\label{subsubsec:fontsearch}

Because there is no description of how to check the available fonts in
the documentation of \pkg{fontspec}, I will describe briefly here.

\XeTeX{} ordinarily search and call fonts with library
\textit{fontconfig}, so |fc-list| will show all the available fonts.
Type the following command in the command line (|cmd| under Windows,
Terminal under Linux):
\begin{frameverb}
  fc-list > fontlist.txt
\end{frameverb}
This will save a list of all the fonts installed into file
\file{fontlist.txt}.

A lot of information is listed with |fc-list|, and it will be long
under Windows with hundreds of fonts installed. For example, this might
be a section of the list:
\begin{frameverb}
  Times New Roman:style=cursiva,kurzíva,kursiv,Πλάγια,Italic,
    Kursivoitu,Italique,Dőlt,Corsivo,Cursief,kursywa,Itálico,Курсив,
    İtalik,Poševno,nghiêng,Etzana
  Times New Roman:style=Negreta cursiva,tučné kurzíva,fed kursiv,
    Fett Kursiv,Έντονα Πλάγια,Bold Italic,Negrita Cursiva,
    Lihavoitu Kursivoi,Gras Italique,Félkövér dőlt,Grassetto Corsivo,
    Vet Cursief,Halvfet Kursiv,Pogrubiona kursywa,Negrito Itálico,
    Полужирный Курсив,Tučná kurzíva,Fet Kursiv,Kalın İtalik,
    Krepko poševno,nghiêng đậm,Lodi etzana
  Times New Roman:style=Negreta,tučné,fed,Fett,Έντονα,Bold,Negrita,
    Lihavoitu,Gras,Félkövér,Grassetto,Vet,Halvfet,Pogrubiona,Negrito,
    Полужирный,Fet,Kalın,Krepko,đậm,Lodia
  Times New Roman:style=Normal,obyčejné,Standard,Κανονικά,Regular,
    Normaali,Normál,Normale,Standaard,Normalny,Обычный,Normálne,Navadno,
    thường,Arrunta
  宋体,SimSun:style=Regular
  黑体,SimHei:style=Normal,obyčejné,Standard,Κανονικά,Regular,Normaali,
    Normál,Normale,Standaard,Normalny,Обычный,Normálne,Navadno,Arrunta
\end{frameverb}
The font name used in \pkg{fontspec} and \pkg{xeCJK} is the part before
the colon. For example,
\begin{ctexexam}
  \setmainfont{Times New Roman}
  \setCJKmainfont{SimSun} % 或者 \setCJKmainfont{宋体}
\end{ctexexam}
might ne used in the document.

For simplicity's sake, options may be added to control the output
format. For example, if we only want the font name for all Chinese
fonts, we can use:
\begin{frameverb}
  fc-list -f "%{family}\n" :lang=zh  > zhfont.txt
\end{frameverb}
In this way, the list of fonts is saved in file \file{zhfont.txt}.\footnote{Because of the encoding issue, the list 
should always be output to a separate file under Windows.} The list
now would be simpler and more readable. For example, here are
pre-installed Chinese fonts under Windows:
\begin{frameverb}
  Arial Unicode MS
  FangSong,仿宋
  KaiTi,楷体
  Microsoft YaHei,微软雅黑
  MingLiU,細明體
  NSimSun,新宋体
  PMingLiU,新細明體
  SimHei,黑体
  SimSun,宋体
\end{frameverb}
To list Japanese or Korean font, |zh| in |:lang=zh| should be changed
to |ja| or |ko| respectively.

Fonts' filenames may also be used in \pkg{fontspec} and \pkg{xeCJK}.
For example, |Simsun| under Windows may also be called with
\begin{frameverb}
  \setCJKmainfont{simsun.ttc}
\end{frameverb}
since it is a TTC font. Related options and syntax for setting a font's
filename is described in detail in the documentation of \pkg{fontspec},
so I will not repeat. There are several Chinese fonts whose name is
irregular and cannot be accesses with font names, then this method can
be used.

\subsection{CJK font specification by block}
\label{subsec:block}

The number of CJK characters is immense, and it is impossible for one 
font to contain all of them. \pkg{xeCJK} allows declaring different
fonts under the same CJK family, and use them to output CJK characters
in different blocks. To do so, the blocks need to be declared in
advance.

\begin{function}[EXP]{\xeCJKDeclareSubCJKBlock}
  \begin{syntax}
    \tn{xeCJKDeclareSubCJKBlock}  \Arg{block} \Arg{block range}
    \tn{xeCJKDeclareSubCJKBlock} * \Arg{block} \Arg{block range}
  \end{syntax}
  \meta{block range} is a list separated by commas, containing either
  a |Unicode| range of CJK characters, or an individual character as
  |Unicode|, such as
\end{function}
  \begin{ctexexam}
  { `中 -> `文 , "3400 -> "4DBF , "5000 -> "7000 , `汉 , `字 , "3500 }
  \end{ctexexam}
  The \meta{block range} being set here should not exceed the
  \hyperlink{CJKcharclass}{CJK character range} set in the source code,
  unless it is necessary (for example, some special fonts that use the PUA region of |Unicode|). The following code
  \begin{ctexexam}
  \xeCJKDeclareSubCJKBlock{SPUA}{ "E400 -> "E4DA , "E500 -> "E5E8 , "E600 -> "E6CE }
  \xeCJKDeclareSubCJKBlock{Ext-B}{ "20000 -> "2A6DF }
  \end{ctexexam}
  declares two sub-block called |SPUA| and |Ext-B|, and creates two
  new options, again |SPUA| and |Ext-B|, in \meta{font features} of CJK
  font setting described in Section~\ref{subsec:fontset}.
  These two new options are analogous to |FallBack| described in
  Section~\ref{subsec:fontset}, and they may be used in font setting.

  For example, the declaration
  \begin{ctexexam}
  \setCJKmainfont[SPUA=SunmanPUA,Ext-B=SimSun-ExtB]{SimSun}
  \end{ctexexam}
  sets |SimSun| as the main font of the document, using |SunmanPUA| for
  |SPUA| sub-block, and using |SimSun-ExtB| for |Ext-B| sub-block.

  \tn{xeCJKDeclareSubCJKBlock} should be used before all declaration
  of CJK font families. If some CJK font family has no setting for
  certain \meta{block}s, the setting for \tn{CJKfamilydefault} will be
  used instead. If one wants to declare a font family without font
  switch between defferent \meta{block}s, option \meta{block}|=*| may
  be used. The starred version resets character classes of punctuation
  besides setting CJK sub-blocks.

\begin{function}{\xeCJKCancelSubCJKBlock}
  \begin{syntax}
    \tn{xeCJKCancelSubCJKBlock}  \Arg{block_1, block_2, ...}
    \tn{xeCJKCancelSubCJKBlock} \Arg{block_1, block_2, ...}
  \end{syntax}
  Cancel the declaration of |CJK| blocks in the document. The starred
  version resets character classes of punctuation.
\end{function}

\begin{function}{\xeCJKRestoreSubCJKBlock}
  \begin{syntax}
    \tn{xeCJKRestoreSubCJKBlock}  \Arg{block_1, block_2, ...}
    \tn{xeCJKRestoreSubCJKBlock} * \Arg{block_1, block_2, ...}
  \end{syntax}
  Restore the declaration of |CJK| blocks in the document. The starred
  version resets character classes of punctuation.
\end{function}

\subsection{Setting of CJK character ranges}

\begin{function}[EXP]{\xeCJKDeclareCharClass}
  \begin{syntax}
    \tn{xeCJKDeclareCharClass}  \Arg{class} \Arg{class range}
    \tn{xeCJKDeclareCharClass} * \Arg{class} \Arg{class range}
  \end{syntax}
  The format of \meta{class range} is the same as that of
  \meta{block range} mentioned in Section~\ref{subsec:block}. The valid
  values of \meta{class} are described in the source code
  (Section~\ref{sec:xeCJK-class-set}). \pkg{xeCJK} has supported all
  CJK characters and punctuations in |Unicode|. Generally, the
  character class should not be modified without reason. The starred command resets character classes of punctuation
  besides setting CJK sub-blocks, in order to make sure that
  punctuations are handled correctly.
\end{function}

\begin{function}[EXP]{\xeCJKResetCharClass}
  Restore the setting of character classes to the default of
  \pkg{xeCJK}.
\end{function}

\begin{function}[EXP]{\xeCJKResetPunctClass}
  Restore the character classes of CJK punctuations.
\end{function}

\begin{function}{\normalspacedchars}
  \begin{syntax}
    \tn{normalspacedchars} \Arg{char list}
  \end{syntax}
  No spacing will be added to either side of the characters in
  \meta{char list}. The default setting is |/|, |\|,  and |-|
  (|U+002D|).
\end{function}

\subsection{Processing of punctuations}

\pkg{xeCJK} adjusts the output width of punctuation characters by
adjusting its side bearings. Currently, for left punctuation (such as
left double quote), \pkg{xeCJK} can only adjust its left side
bearings; for right punctuation (such as comma or right double
quote), \pkg{xeCJK} can only adjust its right side bearings; for
centered punctuation, both side bearings will be adjusted, in order to
keep it centered. The related settings for punctuation can only be made
in the preamble.

\subsubsection{Setting width and side bearings for specific punctuation}
\label{subsec:punct}

The settings here applies to all punctuation styles except |plain|.

\begin{function}[EXP,updated=2013-08-22]{\xeCJKsetwidth}
  \begin{syntax}
    \tn{xeCJKsetwidth}  \Arg{punct list} \Arg{length}
    \tn{xeCJKsetwidth} * \Arg{punct list} \Arg{length}
  \end{syntax}
  \meta{punct list} can consist of one or more punctuations. For example,
  \begin{ctexexam}
  \xeCJKsetwidth{。?}{0.7em}
  \end{ctexexam}
  sets the width of full stop and question mark to |0.7em|. The starred
  version sets the width at the beginning/ending of lines.
\end{function}

\begin{function}[EXP]{\xeCJKsetkern}
  \begin{syntax}
    \tn{xeCJKsetkern} \Arg{punct 1} \Arg{punct 2} \Arg{length}
  \end{syntax}
  \pkg{xeCJK} adjusts the space between two adjacent |CJK| punctuation
  according to punctuation style. This command can be used to modify
  specific cases. For example,
  \begin{ctexexam}
  \xeCJKsetkern{:}{“}{0.3em}
  \end{ctexexam}
  sets the space between a colon and a left double quote as |0.3em|.
\end{function}

\subsubsection{Define punctuation style.}
\label{subsec:punctstyle}

\begin{function}[EXP,updated=2013-08-22]{\xeCJKDeclarePunctStyle}
  \begin{syntax}
    \tn{xeCJKDeclarePunctStyle} \Arg{style} \Arg{options}
  \end{syntax}
  Define a new punctuation style. The existing style of the same name
  will be overwritten. The options can be set will be introduced below.
\end{function}

\begin{function}[EXP,updated=2013-08-22]{\xeCJKEditPunctStyle}
  \begin{syntax}
    \tn{xeCJKEditPunctStyle} \Arg{style} \Arg{options}
  \end{syntax}
  Modify an existing punctuation style.
\end{function}

The following are valid options in the setting. The left column is the
name of option, the middle column is the type of parameter, and the
right column is description. Some options are mutually exclusive, and
there is a matter of precedence: the options of lower precedence is
effective only when that of higher precedence is disabled, which means
setting to |false| for \meta{boolean}, \tn{maxdimen} for
\meta{length}, or |nan| for \meta{real}.

\begin{psopt}
  \PSKeyVal{enabled-global-setting}{boolean}
    Whether to use |PunctWidth| and |PunctBoundWidth| options in
    \tn{xeCJKsetup}, as well as settings of \tn{xeCJKsetwidth} and
    \tn{xeCJKsetkern}. Default is |true|.
\end{psopt}

\begin{psopt}
  \PSKeyVal{fixed-punct-width}{length}
    Set the width of a single punctuation. Default is \tn{maxdimen}.
  \PSKeyVal{fixed-punct-ratio}{real}
    Set the ratio of the output width to the actual width of a single
    punctuation. Default is |1.0|.
  \PSKeyVal{mixed-punct-width}{length}
    Set the width of end-of-sentence punctuation, as defined by
    |KaiMingPunct| in \tn{xeCJKsetup}. Default is the same as
    |fixed-punct-width|.
  \PSKeyVal{mixed-punct-ratio}{real}
    Set the width ratio of end-of-sentence punctuation. Default is the
    same as |fixed-punct-ratio|.
  \PSKeyVal{middle-punct-width}{length}
    Set the width ratio of centered punctuation, as defined by
    |MiddlePunct| in \tn{xeCJKsetup}. Default is the same as
    |fixed-punct-width|.
  \PSKeyVal{middle-punct-ratio}{real}
    Set the width ratio of centered punctuation. Default is the
    same as |fixed-punct-ratio|.
\end{psopt}

The options above sets the fixed width or ratio of punctuation, and
\pkg{xeCJK} calculates the left/right side bearings of punctuation
based on these options. The following options sets the left/right side
bearings of punctuation directly, so the width of different punctuation might be different. The following options would be
effective only when the related options above are disabled. The
precedence decreases from top to bottom.

\begin{psopt}
  \PSKeyVal{fixed-margin-width}{length}
    Set the left/right side bearing of punctuation. default is \tn{maxdimen}.
  \PSKeyVal{fixed-margin-ratio}{real}
    Set the ratio of left/right side bearings of punctuation to the actual side bearing in the font. Default is~|1.0|。
  \PSKeyVal{mixed-margin-width}{length}
    Set the left/right side bearing of end-of-sentence punctuation. Default is the same as |fixed-margin-width|.
  \PSKeyVal{mixed-margin-ratio}{real}
    Set the ratio of left/right side bearings of end-of-sentence punctuation. Default is the same as |fixed-margin-ratio|.
  \PSKeyVal{middle-margin-width}{length}
    Set the left/right side bearing of centered punctuation. Default is the same as |fixed-margin-width|.
  \PSKeyVal{middle-margin-ratio}{real}
    Set the left/right side bearings of centered punctuation. Default is the same as |fixed-margin-width|.
\end{psopt}

The following options sets the width or ratio for punctuation at the beginning or ending of a line.

\begin{psopt}
  \PSKeyVal{bound-punct-width}{length}
    Set the width of punctuation occurring at the beginning or ending of a line. Default is \tn{maxdimen}.
  \PSKeyVal{bound-punct-ratio}{real}
    Set the ratio of output width of punctuation occurring at the beginning or ending of a line to the actual width. Default is |nan|.
  \PSKeyVal{bound-margin-width}{length}
    Set the left/right side bearing of punctuation occurring at the beginning or ending of a line. Default is \tn{maxdimen}.
  \PSKeyVal{bound-margin-ratio}{real}
    Set the ratio of left/right side bearings of punctuation occurring at the beginning or ending of a line to the actual side bearings. Default is |0|.
  \PSKeyVal{enabled-hanging}{boolean}
    Whether or not punctuation hanging is allowed, when the width calculated from options above is smaller than the
  actual glyph width. Default is |false|。
\end{psopt}

\begin{psopt}
  \PSKeyVal{add-min-bound-to-margin}{boolean}
    Whether or not the less of left/right side bearing should be added to the result calculated. Does not affect centered punctuation.
    Default is |false|.
\end{psopt}

\begin{psopt}
  \PSKeyVal{optimize-margin}{boolean}
    When using the options above, the left/right side bearing of a
    punctuation might be larger than the original. If the option is set
    to |true|, the original side bearing is used; when the left/right side bearing is smaller than the side bearing of the other side, the value of the other side is used. Default is |false|.
\end{psopt}

\begin{psopt}
  \PSKeyVal{margin-minimum}{length}
    Set the minimum left/right side bearing of punctuation. When the
    result calculated is smaller than this value, then it is used.
    Default is |0pt|.
\end{psopt}

The following options adjust the space between two adjacent
punctuation. They are mutually exclusive, and precedence decreases from
top to bottom.

\begin{psopt}
  \PSKeyVal{enabled-kerning}{boolean}
    Whether or not spacing is adjusted for two adjacent punctuation.
    If it is set to |false|, all punctuation are output as the original
    output width. Default is |true|.
  \PSKeyVal{min-bound-to-kerning}{boolean}
    Whether or not to use the more one between the lesser of side
    bearings of the former punctuation and that of the latter one as
    the spacing between two punctuation. Default is |false|.
  \PSKeyVal{kerning-total-width}{length}
    Set the total width of two punctuation. \pkg{xeCJK} will then
    calculate the spacing between two punctuation automatically.
    Default is \tn{maxdimen}.
  \PSKeyVal{kerning-total-ratio}{real}
    Set the ratio of total output width to the actual width. Default
    is |0.75|.
  \PSKeyVal{same-align-margin}{length}
    Set the spacing between two punctuation at the same side. Default
    is \tn{maxdimen}.
  \PSKeyVal{same-align-ratio}{real}
    Set the ratio of spacing between two punctuation at the same side
    to the actual spacing. Default is |nan|.
  \PSKeyVal{different-align-margin}{length}
    Set the spacing between two punctuation at different sides.
    Default is \tn{maxdimen}.
  \PSKeyVal{different-align-ratio}{real}
    Set the ratio of spacing between two punctuation at different
    sides to the actual spacing. Default is |nan|.
  \PSKeyVal{kerning-margin-width}{length}
    Set the spacing between two adjacent punctuation. Default is
    \tn{maxdimen}.
  \PSKeyVal{kerning-margin-ratio}{real}
    Set the ratio of spacing between two punctuation to the actual
    spacing. Default is |1.0|.
\end{psopt}

\begin{psopt}
  \PSKeyVal{optimize-kerning}{boolean}
    The spacing between punctuation calculated based on the option
    above may be less than that calculated based on
    |min-bound-to-kerning|. When it is set to |true|, then the value
    from |min-bound-to-kerning| is used instead. Default is |false|.
\end{psopt}

\begin{psopt}
  \PSKeyVal{kerning-margin-minimum}{length}
    Set the minimum spacing between two punctuation. When the value
    calculated based on options above is less than its value, its
    value is used instead. Default is |0pt|.
\end{psopt}

In fact, the default setting of \pkg{xeCJK} is equivalent to the
full-width (|quanjiao|) style. The options used above may be used to
define new punctuation styles. For example,
\begin{ctexexam}
  \xeCJKDeclarePunctStyle { mine }
    {
      fixed-punct-ratio       = nan ,
      fixed-margin-width      = 0 pt ,
      mixed-margin-width      = \maxdimen ,
      mixed-margin-ratio      = 0.5 ,
      middle-margin-width     = \maxdimen ,
      middle-margin-ratio     = 0.5 ,
      add-min-bound-to-margin = true ,
      bound-punct-width       = 0 em ,
      enabled-hanging         = true ,
      min-bound-to-kerning    = true ,
      kerning-margin-minimum  = 0.1 em
    }
\end{ctexexam}
defines a new punctuation style called |mine|, and can be used with
\begin{frameverb}
  \xeCJKsetup{PunctStyle=mine}
\end{frameverb}
in the preamble. The meaning is:
\begin{itemize}
  \item Use the lesser of the side bearings of punctuation as both of
    its side bearings; for end-of-sentence or centered punctuation, half
    of its actual side bearing is added;
  \item When a punctuation occurs at the beginning or ending of a
    line, it has a width of zero, and hanging is allowed;
  \item Use the lesser of actual side bearings of two adjacent
    punctuation as the spacing between then, with a minimum of |0.1em|.
\end{itemize}
Another example:
\begin{ctexexam}
  \xeCJKEditPunctStyle { hangmobanjiao } { enabled-global-setting = false }
\end{ctexexam}
This disables the settings of \tn{xeCJKsetkern} and so on to
|hangmobanjiao| punctuation style.

\subsection{Usage of \pkg{xeCJKfntef}}

\pkg{xeCJK} bundle includes a package called \pkg{xeCJKfntef}, which
can be used to put dots (\CJKunderdot{汉字加点}) and breakable underline
under CJK characters. It is the replacement of \pkg{CJKfntef} under
\XeLaTeX{}, and the user interface is the same.

\pkg{xeCJKfntef} is based on package \package{ulem}, and some
extensions are made besides \pkg{ulem} commands.

\begin{function}[updated=2014-11-04]
 {\CJKunderline,\CJKunderdblline,\CJKunderwave,\CJKsout,\CJKxout}
  \begin{syntax}
    \tn{CJKunderline} [*] [-] \oarg{options} \Arg{text}
  \end{syntax}
  \begin{SideBySideExample}[frame=single,numbers=left,xrightmargin=.35\linewidth,gobble=4]
    \CJKunderline{虚室生白,吉祥止止}\\
    \CJKunderdblline{虚室生白,吉祥止止}\\
    \CJKunderwave{虚室生白,吉祥止止}\\
    \CJKsout{虚室生白,吉祥止止}\\
    \CJKxout{虚室生白,吉祥止止}
  \end{SideBySideExample}
\end{function}

\csappto{NoHighlight@Attributes}{\catcode37=14\relax}

\begin{Example}[frame=single,numbers=left,gobble=2]
  \CJKunderline-{南朝}\CJKunderline-{梁}\CJKunderline-{劉勰}%
  \CJKunderwave-{文心雕龍}\CJKunderwave-{養氣}\\
  \CJKunderline*[thickness=1pt, hidden=true]{瞻彼阕者,虚室生白,吉祥止止}
\end{Example}

\begin{function}[updated=2014-11-04]{\CJKunderdot}
  \begin{syntax}
    \tn{CJKunderdot} \oarg{options} \Arg{text}
  \end{syntax}
  Add dots under CJK characters, can be used in conjunction with the
  underline comment above. For example:

  \begin{SideBySideExample}[frame=single,numbers=left,xrightmargin=.35\linewidth,gobble=4]
    \CJKunderline{虚室生白,\CJKunderdot{吉祥}止止}\\
    \CJKunderdot{虚室生白,\CJKunderline{吉祥}止止}
  \end{SideBySideExample}
\end{function}

\bigskip

For the six commands above, \pkg{xeCJKfntef} offers some options to
set the position or color of dots and lines. They can either be set
using \tn{xeCJKsetup} in advance, or during using.

\begin{function}[added=2014-11-04]{skip}
  \begin{syntax}
    \tn{xeCJKsetup} \{ underline/skip = \meta{\TTF} \}
    \tn{xeCJKsetup} \{ underline = \{ skip = \meta{\TTF} , ... \} \}
  \end{syntax}
  By default, the underline will skip punctuation. This feature can be
  disabled by setting the option to \texttt{false}. Adding |*| after
  the command has the same effect.
\end{function}

\begin{function}{subtract}
  When it is set to \texttt{true}, the underline will be
  slightly shortened, so that adjacent underlines will not be
  connected. This can be used for marking proper nouns or book titles
  in pre-modern writing. Adding |-| after the command has the same
  effect.
\end{function}

\begin{function}{hidden}
  When it is set to \texttt{true},the text itself will be hidden,
  and only the underline is drawn.
\end{function}

\begin{function}{format}
  \begin{syntax}
    \tn{xeCJKsetup} \{ underline/format = \tn{color}\{red\} \}
    \tn{xeCJKsetup} \{ underwave = \{ format = \tn{color}\{red\}, ... \} \}
  \end{syntax}
  Set the format of line or dot, for example, color.
\end{function}

\begin{function}[added=2016-06-03]{textformat}
  Set the fornat of the underlined or dotted text. For example:\smallskip
  \begin{Example}[frame=single,numbers=left,gobble=4]
    \CJKunderline[textformat=\color{red}]{虚室生白,吉祥止止}\\
    \CJKunderdot[textformat=\bfseries, format=\color{blue}]{虚室生白,吉祥止止}
  \end{Example}
\end{function}

\begin{function}{symbol}
  Set the symbol of \tn{CJKunderwave} or \tn{CJKunderdot}.
\end{function}

For example, the symbol of \tn{CJKunderwave} will not change when font
size changes, so it does not look good under small sizes. We can modify
it to change size as font size changes:

\begin{SideBySideExample}[frame=single,numbers=left,xrightmargin=.35\linewidth]
  \xeCJKsetup{%
    underwave/symbol=
      \fontsize{0.5em}{0pt}%
      \fontencoding{U}\fontfamily{lasy}\selectfont
      \char 58\relax}
  \footnotesize
  \CJKunderwave{瞻彼阕者,虚室生白,吉祥止止}
\end{SideBySideExample}

\begin{function}{thickness}
  Set the thickness of lines used in \tn{CJKunderline},
  \tn{CJKunderdblline} and \tn{CJKsout}. Default is \tn{ULthickness}.
\end{function}

\begin{function}{depth}
  Set the depth of lines or dots (distance from the baseline to the
  top of the line or dot). Default is \texttt{0.2em}.
\end{function}

\begin{function}{boxdepth}
  \tn{CJKunderdot} might affect line spacing, which can be adjusted
  by this option. If it is not desirable that \tn{CJKunderdot} affect
  line spacing, this option can be set to \texttt{0pt}.
\end{function}

\begin{function}{sep}
  Set the separation between lines/dots when \tn{CJKunderdot} is used 
  along with \tn{CJKunderline}, \tn{CJKunderdblline} or 
  \tn{CJKunderwave}.
\end{function}

\begin{function}{gap}
  Set the distance between two lines in \tn{CJKunderdblline}. Default
  is \texttt{1.1pt}。
\end{function}

\begin{function}{height}
  Set the height of cross-out line \tn{CJKsout} (distance from the center of line to the baseline). Default is \texttt{0.35em}.

  For example, we can set the thickness and color of \tn{CJKsout}, so it
  will look like highlighting:\smallskip

  \begin{Example}[frame=single,numbers=left,gobble=4]
    \CJKsout*[thickness=2.5ex, format=\color{yellow}]{瞻彼阕者,虚室生白,吉祥止止}
  \end{Example}
\end{function}

\medskip

\pkg{xeCJKfntef} also offers \tn{CJKunderanyline} and
\tn{CJKunderanysymbol} to define styles of underline and symbols.

\begin{function}[updated=2014-11-07]{\CJKunderanyline}
  \begin{syntax}
    \tn{CJKunderanyline} [*] [-] \oarg{options} \Arg{depth} \Arg{underline} \Arg{text}
  \end{syntax}
  \pkg{xeCJKfntef} first put \meta{underline} into a box called
  \tn{xeCJKfntefbox}, move it down by \meta{depth}, and use it for
  filling. Valid \meta{options} are \texttt{textformat},
  \texttt{skip}, \texttt{hidden}, \texttt{subtract}, \texttt{sep} and
  \texttt{boxdepth}. The initial value of \texttt{sep} and
  \texttt{boxdepth} are empty, which means that they are disabled. They
  can be set using \texttt{ulem} in \tn{xeCJKsetup}.
\end{function}

For example, this can also be used for highlighting: \smallskip

\begin{Example}[frame=single,numbers=left,gobble=2]
  \CJKunderanyline*{0.5ex}{\color{yellow}\rule{2pt}{2.5ex}}{虚室生白,吉祥止止}
\end{Example}

\begin{function}[updated=2014-11-04]{\CJKunderanysymbol}
  \begin{syntax}
    \tn{CJKunderanysymbol} \oarg{options} \Arg{depth} \Arg{symbol} \Arg{text}
  \end{syntax}
  \pkg{xeCJKfntef} put \meta{symbol} into a box called
  (\tn{xeCJKfntefbox}). \meta{depth} us used to set the box's depth
  (distance from the baseline to the top of the box). Valid 
  \meta{options} are \texttt{textformat}, \texttt{sep} and
  \texttt{boxdepth},with the same meaning as in \tn{CJKunderdot}.
\end{function}

For example, the code below adds triangle under Chinese characters:\smallskip

\begin{Example}[frame=single,numbers=left,gobble=2]
  \CJKunderanysymbol[sep=0.1em]{0.2em}{\tiny$\triangle$}
    {瞻彼阕者,虚室生白,\CJKunderline{吉祥止止}}
\end{Example}

\begin{function}[updated=2014-11-07]{\xeCJKfntefon}
  \begin{syntax}
    \tn{xeCJKfntefon} [*] [-] \oarg{options}
  \end{syntax}
  The same as \tn{ULon} offered in package \pkg{ulem}, with the
  extension of |*| 和 |-|. Valid \meta{options} are
  \texttt{textformat}, \texttt{skip}, \texttt{hidden} and
  \texttt{subtract}. These options are also effective to command
  \tn{uline} and so on in \pkg{ulem}, and it requires setting
  \texttt{ulem} in \tn{xeCJKsetup}. For example:\smallskip

  \begin{Example}[frame=single,numbers=left,gobble=4]
    \xeCJKsetup{ulem={textformat=\bfseries\color{red}, skip=true}}
    \uline{虚室生白,吉祥止止}
  \end{Example}
\end{function}

\medskip

In addition, \pkg{xeCJKfntef} also has environments that allows Chinese
characters to spread across the set width: \env{CJKfilltwosides} and \env{CJKfilltwosides*}。

\begin{function}[updated=2014-11-04]{CJKfilltwosides}
  \begin{syntax}
    \tn{begin}\{CJKfilltwosides\} \oarg{position} \Arg{width}
      text\verb=\\=
      text
    \tn{end}\{CJKfilltwosides\}
  \end{syntax}
  The content in the environment is pun into a \tn{vbox}. The
  optional parameter \meta{position} sets the baseline position of the
  box to as \texttt{t}op, \texttt{c}enter or \texttt{b}ottom; default
  is \texttt{c}.
  \meta{宽度} sets the width of the box.
  \env{CJKfilltwosides*} is different from \env{CJKfilltwosides}, in that it will use the natural width of the box when \meta{width} is zero, negative or smaller than the natural width. For example,
\end{function}

\begin{SideBySideExample}[frame=single,numbers=left,xrightmargin=.5\linewidth,gobble=2]
  \begin{CJKfilltwosides}{.8\linewidth}
    瞻彼阕者,\\
    虚室生白,吉祥止止
  \end{CJKfilltwosides}
\end{SideBySideExample}

\begin{SideBySideExample}[frame=single,numbers=left,xrightmargin=.5\linewidth,gobble=2]
  \begin{CJKfilltwosides*}{0pt}
    瞻彼阕者,\\
    虚室生白,吉祥止止
  \end{CJKfilltwosides*}
\end{SideBySideExample}

\subsection{Others}
\label{subsec:others}

\begin{function}[updated=2013-11-16]{\xeCJKVerbAddon,\xeCJKOffVerbAddon}
  Adjust the spacing between CJK characters, so that they the space
  occupied equals two spaces in the Western monospace font. If the
  width of two spaces is smaller than the normal width of one CJK character, then the CJK characters will be shrunk. This is useful
  for situations like code alignment. It need to be noticed that \tn{xeCJKVerbAddon} modifies \pkg{xeCJK} internals greatly, and line
  breaking between CJK characters will be forbidden if it is enabled.
  Therefore, it should never be used alone, but should be put into a
  a group or environment to keep the modifications local. Otherwise,
  it will be disabled.
  
  The option may be used along with other options about verbatim/code,
  such as \texttt{formatcom} option in package\package{fancyvrb}. The
  Western font being used should be monospaced to keep alignment. If
  the Western font (including size) is changed, then
  \tn{xeCJKVerbAddon} should be execute again to re-calculate the
  amount of spacing. \tn{xeCJKOffVerbAddon} is used in the group or
  environment to locally disable \tn{xeCJKVerbAddon}. Because
  \package{listings} use its own mechanism of code alignment, 
  \tn{xeCJKVerbAddon} is disabled in its code environments.
\end{function}

\begin{function}[added=2012-12-03]{\xeCJKnobreak}
  \begin{syntax}
    ……汉字。\tn{xeCJKnobreak}\tn{footnote}\{脚注\}
  \end{syntax}
  \tn{xeCJKnobreak} is used after a full-width punctuation to avoid
  line breaking. It is unnecessary if |CheckFullRight| described before
  is invalid.
\end{function}

\begin{function}[added=2013-11-09]{\xeCJKShipoutHook}
  \pkg{xeCJK} has some special settings (dots under Chinese
  characters, line breaking in \env{verbatim} or \env{lstlisting}
  environment) that may affect \TeX{}'s output routine such as header or
  footer. \tn{xeCJKShipoutHook} is used to restore the ordinary settings. \pkg{xeCJK} has already processed the header and footer,
  while manual calling is still necessary in other situations need.
  For example, when \pkg{eso-pic} or \pkg{atbegshi} is used for
  watermark, and the special situations described above are used in the
  document, then \tn{xeCJKShipoutHook} should used at the beginning of
  \tn{AtBeginShipout}.
\end{function}

\section{Known Issues and Compatibility}

\XeTeX{} set the \tn{catcode} of all CJK ideographs to 11 in
\file{unicode-letters.tex}. Therefore, Chinese characters can be used
in control sequences, while a space should be used between a control
sequence and a Chinese character. Otherwise, message ``\texttt{! Undefined control sequence.}'' will appear.

\pkg{xeCJK} uses and redefines some commands from package \pkg{CJK},
such as \tn{CJKfamily}, \tn{CJKsymbol} and \tn{CJKglue}. \pkg{xeCJK}
does not require \pkg{CJK}, and \pkg{xeCJK} forbids loading \pkg{CJK}
after it. However, \pkg{CJKnumb} may be loaded \textit{after}
\pkg{xeCJK} to support Chinese numerals; \package{zhnumber} can also
be used.

\pkg{xeCJK} have some processes that allows direct Unicode in package
\package{listings}, so Chinese characters may be used directly in \texttt{listings} environment, and \texttt{escapechar} is unnecessary.

The newer version (\texttt{3.x}) of \pkg{xeCJK} is written completely
in \LaTeXiii{} syntax. \LaTeXiii{} abandons the concept of \tn{outer}
macro, so related tools using the macro may face problems. In the
current implementation, an \tn{outer} after a CJK character will create
an error like
\begin{frameverb}
  ! Forbidden control sequence found while scanning use of \use_i:nn
\end{frameverb}
An example of this is \tn{cprotect} command offered in package
\package{cprotect}. It is defined as
\begin{frameverb}
  \outer\long\def\cprotect{\icprotect}
\end{frameverb}
so \tn{icprotect} can be used in place of \tn{cprotect}. In fact, when
\pkg{cprotect} is being loaded, \pkg{xeCJK} will use
\begin{frameverb}
  \let\cprotect\icprotect
\end{frameverb}
to cancel the outer macro limitation. \tn{cprotect} However, the
specialty of \tn{cprotect} means that it should only be used outerly,
or never as a parameter of another macro. Other situations involving
\tn{outer} can be resolved by placing \tn{relax} in front of it.

\pkg{xeCJK} relies on \tn{XeTeXinterchartoks} mechanism of \XeTeX{}, and
there might be conflict with other packages using the same mechanism
like \pkg{polyglossia} and \pkg{xesearch}. Although \pkg{xeCJK} tried
to deal with these conflicts, attention should still be made when they
are used together.

\end{document}

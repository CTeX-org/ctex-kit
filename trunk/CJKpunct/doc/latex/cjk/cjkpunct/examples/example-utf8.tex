% -*- coding: utf-8 -*-
\documentclass{article}
\usepackage{CJKutf8}
\usepackage{CJKpunct}


\begin{document}

\begin{CJK*}{UTF8}{gbsn}
\parindent 2em


\long\def\sometexts{
话说宝玉来至院外,就有跟贾政的几个小厮上来拦腰抱住, 都说:“今儿亏我们, 老爷才喜欢,老太太打发人出来问了几遍,都亏我们回说喜欢,不然,若老太太叫
你进去,就不得展才了。人人都说,你才那些诗比世人的都强。今儿得了这样的彩头。 该赏我们了。”宝玉笑道:“每人一吊钱。”众人道:“谁没见那一吊钱!把这荷包赏了罢。” 说着,一个上来解荷包, 那一个就解扇囊,不容分说,将宝玉所佩之物尽行解去。 又道:“好生送上去,罢。”一个抱了起来,几个围绕,送至贾母二门前。
那时贾母已命人看了几次。 众奶娘丫鬟跟上来,见过贾母,知不曾难为着他,心中自是欢喜。

少时袭人倒了茶来,见身边佩物一件无存,
因笑道:“带的东西又是那起没脸的东西们解了去了。”林黛玉听说,走来瞧瞧,
果然一件无存,因向宝玉道:“我给的那个荷包也给他们了?你明儿再想我的东西,
可不能够了!”说毕,赌气回房,将前日宝玉所烦他作的那个香袋儿——才做了一
半——赌气拿过来就铰。 宝玉见他生气,便知不妥, 忙赶过来,早剪破了。宝玉已见过这香囊,虽尚未完,却十分精巧,费了许多工夫。
今见无故剪了, 却也可气。因忙把衣领解了,从里面红袄襟上将黛玉所给的那荷包解了下来, 递与黛玉瞧道:“你瞧瞧,这是什么!我那一回把你的东西给人了?”
林黛玉见他如此珍重,带在里面,可知是怕人拿去之意,因此又自悔莽撞,
未见皂白,就剪了香袋。因此又愧又气,低头一言不发。宝玉道:“你也不
用剪,我知道你是懒待给我东西。我连这荷包奉还,何如?”说着,掷向他怀
中便走。黛玉见如此,越发气起来,声咽气堵,又汪汪的滚下泪来,拿起荷包
来又剪。宝玉见他如此,忙回身抢住,笑道:“好妹妹,饶了他罢!” 黛玉将剪子一摔,拭泪说道:“你不用同我好一阵歹一阵的,要恼,就撂开手。
这当了什么。”说着,赌气上床,面向里倒下拭泪。禁不住宝玉上来“妹妹”长“妹妹”短赔不是。

前面贾母一片声找宝玉。众奶娘丫鬟们忙回说:“在林姑娘房里呢。”贾母听
说道:“好,好,好!让他姊妹们一处顽顽罢。才他老子拘了他这半天,让他开
心一会子罢。只别叫他们拌嘴, 不许扭了他。”众人答应着。黛玉被宝玉缠不过,只得起
来道:“你的意思不叫我安生,我就离了你。”说着往外就走。宝玉笑道:“你到
那里,我跟到那里。”一面仍拿起荷包来带上,黛玉伸手抢道:“你说不要了,这
会子又带上,我也替你怪臊的!”说着,“嗤”的一声又笑了。宝玉道:“好妹妹,
明儿另替我作个香袋儿罢。”黛玉道:“那也只瞧我高兴罢了。”一面说,一面二人出
房,到王夫人上房中去了,可巧宝钗亦在那里。

\hfill
——曹雪芹《红楼梦》\unkern\unkern
}


\long\def\showexample#1{\par------ #1 ------\par \punctstyle{#1}\sometexts}


\showexample{quanjiao}


\showexample{banjiao}

\showexample{hangmobanjiao}

\showexample{CCT}

\showexample{kaiming}

 \showexample{plain}




\end{CJK*}


\end{document}


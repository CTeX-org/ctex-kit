%%
%% This is file `example-CJKfntef.tex',
%% generated with the docstrip utility.
%%
%% The original source files were:
%%
%% xeCJK.dtx  (with options: `example-CJKfntef')
%% 
%%  Version 2.2.10 (2-May-2009)
%% 
%%  Copyright (C) Wenchang Sun <sunwch@hotmail.com>
%% 
%%  This file may be distributed and/or modified under the
%%  conditions of the LaTeX Project Public License, either version 1.3
%%  of this license or (at your option) any later version.
%%  The latest version of this license is in
%%    http://www.latex-project.org/lppl.txt
%%  and version 1.3 or later is part of all distributions of LaTeX
%%  version 2005/12/01 or later.
%% 


\documentclass[11pt]{article}
\textheight 220mm
\textwidth 150mm
\oddsidemargin 0pt
\evensidemargin 0pt
\usepackage[slantfont,boldfont]{xeCJK}
\usepackage{xcolor}
\usepackage{CJKfntef}

\begin{document}
\setCJKmainfont{Bitstream CyberCJK}% 设置缺省中文字体
\setCJKmonofont{Bitstream CyberCJK}% 设置缺省中文字体

\baselineskip 16pt
\parindent 2em

 \section{举例}
\begin{verbatim}
标点。
\end{verbatim}

\CJKtextspaces\CJKmathspaces

汉字Chinese数学$x=y$空格

汉字 Chinese 数学 $x=y$ 空格

\CJKunderline{汉字Chinese数学$x=y$加下划线,可以\CJKunderdot{同时加点}。}

\CJKunderline{汉字 Chinese 数学 $x=y$ 加下划线,可以\CJKunderdot{同时加点}。}

\CJKunderline*{汉字加下划线,可以\CJKunderdot{同时加点}。}

\CJKunderdot{汉字加点,可以\CJKunderline{同时加下划线}。}

\end{document}
\endinput
%%
%% End of file `example-CJKfntef.tex'.

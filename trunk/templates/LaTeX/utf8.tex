% latex-chinese-template v0.2   http://ctex-kit.googlecode.com/
%
% See comments at the end of this file for more descriptions.

% Change winfonts to cjkfonts to use arphic fonts.
% Change winfonts to adobefonts to use Adobe fonts with xelatex.
% See http://tug.ctan.org/tex-archive/language/chinese/ctex/doc/
%  or http://ctex-kit.googlecode.com/svn/trunk/ctex/doc/
%  for details about options of ctex macro package.

\documentclass[winfonts,UTF8,cs4size,a4paper,fancyhdr,fntef]{ctexart}
\ifxetex    % xelatex
  \usepackage[xetex]{hyperref}
\else
  \ifpdf    % pdflatex
    \usepackage[pdftex,unicode]{hyperref}
  \else     % dvipdfmx or dvips
    \usepackage[dvipdfmx,unicode]{hyperref}
    %\usepackage[ps2pdf,unicode]{hyperref}
  \fi
\fi

\ifxetex\else\ifpdf\else
  % pdftex 3.1415926-1.40.10-2.2 has trouble with it
  \InputIfFileExists{zhwinfonts.tex}{}{}
\fi\fi

\begin{document}

\title{\LaTeX~中文文档模板}
\author{作者}

\maketitle
\tableofcontents

\section{前言}

朱镕基,你好,CTeX~文档类。

\section{正文}

世界你好!

\section{后记}

非常好!

\end{document}

%%%%%%%%%%%%%%%%%%% END %%%%%%%%%%%%%%%%%%%%%%%%%%%%%%%%%%%%%%%%%%%%%%%%

% How to compile:
%   install TeXLive 2009 as described below;
%   if you use adobefonts with xelatex, put Adobe fonts into ~/.fonts (Linux) or C:\WINDOWS\fonts (Windows);
%   else put sim*.ttf, sim*.ttc  into ~/.fonts and $TEXMFHOME/fonts/truetype/sim/ ;
%   See http://code.google.com/p/ctex-kit/wiki/SimpleChineseTemplates and Makefile for detailed building instructions.
%
%%%%%%%%%%%%%%%%%%% ENVIRONMENT %%%%%%%%%%%%%%%%%%%%%%%%%%%%%%%%%%%%%%%%

% Tested with TeXLive 20090909 on Debian Squeeze: (use `tlmgr show xxx` for details)
%  xetex        r14602  3.1415926-2.2-0.9995.2
%  pdftex       r14549  3.1415926-1.40.10-2.2
%  ctex         r14619  0.93    (buggy, use r167 at http://ctex-kit.googlecode.com/)
%  zhmetrics    r14469          (buggy, use r167 at http://ctex-kit.googlecode.com/)
%  cjk          r15155  4.8.2
%  xecjk        r14567  2.3.5   (outdated, use r167 at http://ctex-kit.googlecode.com/)
%  arphic       r13822
%
% Way to install TeXLive 2009:
%  install-tl -repository http://ftp.ctex.org/mirrors/texlive/tlnet/  -profile texlive.profile
%
% Content of texlive.profile(without the leading "%" and white spaces):
%   # texlive.profile written on Fri Sep  4 07:00:20 2009 UTC
%   # It will NOT be updated and reflects only the
%   # installation profile at installation time.
%   selected_scheme scheme-basic
%   TEXDIR ~/texlive/2009
%   TEXDIRW ~/texlive/2009
%   TEXMFHOME ~/texmf
%   TEXMFLOCAL ~/texlive/texmf-local
%   TEXMFSYSCONFIG ~/texlive/2009/texmf-config
%   TEXMFSYSVAR ~/texlive/2009/texmf-var
%   binary_i386-linux 1
%   collection-basic 1
%   collection-fontsrecommended 1
%   collection-langcjk 1
%   collection-langlatin 1
%   collection-latex 1
%   collection-latexrecommended 1
%   collection-xetex 1
%   from_dvd 0
%   option_desktop_integration 1
%   option_doc 0
%   option_file_assocs 1
%   option_fmt 1
%   option_letter 0
%   option_path 0
%   option_post_code 1
%   option_src 0
%   option_sys_bin /usr/local/bin
%   option_sys_info /usr/local/info
%   option_sys_man /usr/local/man
%   option_w32_multi_user 1
%   option_write18_restricted 1

% View the generated pdf on Debian Squeeze(20090909) with:
%  xpdf (3.02-1.4+lenny1) + xpdf-chinese-simplified (20040727-1, depends cmap-adobe-gb1 (0+20051207-5))
%  evince (2.26.2-2) + poppler-data (0.2.1-5)
% !!! Remember installing poppler-data because it contains data about cmap-adobe-*.

%
% Hope these information useful for TeX newbies :-)
%   Liu Yubao <yubao.liu at gmail dot com>
%
% ChangeLog:
%   2009-09-11  Liu Yubao
%       * initial version, v0.1
%
%   2009-09-13  Liu Yubao
%       * use zhwinfonts to embed truetype fonts with pdflatex and dvipdfmx
%       * set hyperref properly for different encodings and TeX engines to avoid garbled bookmarks
%       % release v0.2
%


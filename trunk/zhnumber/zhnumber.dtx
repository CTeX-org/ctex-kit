% \iffalse meta-comment
% !TEX program  = XeLaTeX
%<*internal>
\iffalse
%</internal>
%<*readme>
Introduction
------------
The zhnumber package provides commands to typeset Chinese representations of
numbers. The main difference between this package and 'CJKnumb' is that commands
provided by this package is expandable in the proper way. So, it seems that
zhnumber is a good alternative to CJKnumb package.

It may be distributed and/or modified under the conditions of the
LaTeX Project Public License (LPPL), either version 1.3c of this license or
(at your option) any later version. The latest version of this license is in

   http://www.latex-project.org/lppl.txt

and version 1.3 or later is part of all distributions of LaTeX version
2005/12/01 or later.

This work has the LPPL maintenance status "maintained".
The Current Maintainer of this work is Qing Lee.

This work consists of the file  zhnumber.dtx,
          and the derived files zhnumber.pdf,
                                zhnumber.sty,
                                zhnumber-utf8.cfg,
                                zhnumber-gbk.cfg,
                                zhnumber-big5.cfg,
                                zhnumber.ins and
                                README (this file).

Basic Usage
-----------
The package provides the following macros:

  \zhnumber{number}
    Convert `number' to a full Chinese representation.

  \zhnum{counter}
    Similar to \arabic{counter}, but representation of 'counter' as Chinese numerals.

  \zhdigits{number}
  \zhdigits*{number}
    Handle `number' as a string of digits and convert each of them into the
    corresponding Chinese digit. The starred version uses the Chinese circle glyph
    for digit zero; the unstarred version uses the traditional glyph.

You can read the package manual (in Chinese) for more detailed explanations.

Author
------
Qing Lee
Email: sobenlee@gmail.com

If you are interested in the process of development you may observe

    http://code.google.com/p/ctex-kit/

Installation
------------
The package is supplied in dtx format and as a pre-extracted zip file,
zhnumber.tds.zip. The later is most convenient for most users: simply
unzip this in your local texmf directory and run texhash to update the
database of file locations. If you want to unpack the dtx yourself, please
ensure that the "iconv" program is installed and working properly, then
running "xetex -shell-escape zhnumber.dtx" will extract the package whereas
"xelatex zhnumber.dtx" will typeset the documentation.

The package requires LaTeX3 support as provided in the l3kernel and l3packages
bundles. Both of these are available on CTAN as ready-to-install zip files.
Suitable versions are available in the latest version of MiKTeX and TeX Live
(updating the relevant packages online may be necessary).

To compile the documentation without error, you will need the xeCJK package
and some specific Chinese Simplified fonts (TrueType or OpenType).
%</readme>
%<*internal>
\fi
\begingroup
  \edef\tempa{\fmtname}
  \edef\tempb{plain}
\expandafter\endgroup
\ifx\tempa\tempb
\csname fi\endcsname
%</internal>
%<*install>

\input l3docstrip.tex
\keepsilent
\askforoverwritefalse
\preamble

   Copyright (C) 2012-2014 by Qing Lee <sobenlee@gmail.com>
--------------------------------------------------------------------------
   This work may be distributed and/or modified under the
   conditions of the LaTeX Project Public License, either version 1.3
   of this license or (at your option) any later version.
   The latest version of this license is in
     http://www.latex-project.org/lppl.txt
   and version 1.3 or later is part of all distributions of LaTeX
   version 2005/12/01 or later.

   This work has the LPPL maintenance status "maintained".
   The Current Maintainer of this work is Qing Lee.

\endpreamble
\postamble

   This package consists of the file  zhnumber.dtx,
                and the derived files zhnumber.pdf,
                                      zhnumber.sty,
                                      zhnumber-utf8.cfg,
                                      zhnumber-gbk.cfg,
                                      zhnumber-big5.cfg,
                                      zhnumber.ins and
                                      README.
\endpostamble

\ifnum\shellescape=1 \else
  \errmessage{
    Shell escape is disabled. Please use ^^J^^J
    xetex -shell-escape \jobname.ins(dtx)^^J}
  \expandafter\endbatchfile
\fi

\generate
  {
    \usedir{source/latex/zhnumber}
    \file{zhnumber.ins}       {\from{\jobname.dtx}{install}}
    \usedir{tex/latex/zhnumber}
    \file{zhnumber.sty}       {\from{\jobname.dtx}{package}}
    \usedir{tex/latex/zhnumber/config}
    \file{zhnumber-utf8.cfg}  {\from{\jobname.dtx}{config,utf8}}
    \file{zhnumber-big5.cfg}  {\from{\jobname.dtx}{config,big5}}
    \file{zhnumber-gbk.cfg}   {\from{\jobname.dtx}{config,gbk}}
    \nopreamble\nopostamble
    \usedir{doc/latex/zhnumber}
    \file{README.txt}         {\from{\jobname.dtx}{readme}}
  }

\immediate\write18{iconv -f utf-8 -t big-5 zhnumber-big5.cfg > zhnumber-big5.temp}
\immediate\write18{iconv -f utf-8 -t gbk zhnumber-gbk.cfg > zhnumber-gbk.temp}
\immediate\write18{mv -f zhnumber-big5.temp zhnumber-big5.cfg}
\immediate\write18{mv -f zhnumber-gbk.temp zhnumber-gbk.cfg}

\endbatchfile
%</install>
%<*internal>
\fi
%</internal>
%
%<*driver|package>
\NeedsTeXFormat{LaTeX2e}
\RequirePackage{expl3}
%</driver|package>
\GetIdInfo$Id$
%<*driver>
  {zhnumber source file}
\ProvidesExplFile{\ExplFileName.\ExplFileExtension}
%</driver>
%<package>  {Typesetting numbers with Chinese glyphs}
%<config&utf8>  {Chinese numerals with UTF8 encoding}
%<config&big5>  {Chinese numerals with Big5 encoding}
%<config&gbk>  {Chinese numerals with GBK encoding}
%<package>\ProvidesExplPackage
%<config>\ProvidesExplFile
%<package>  {\ExplFileName}
%<config&utf8>  {\ExplFileName-utf8.cfg}
%<config&big5>  {\ExplFileName-big5.cfg}
%<config&gbk>  {\ExplFileName-gbk.cfg}
  {\ExplFileDate}{2.0}{\ExplFileDescription}
%<*driver>
\ExplSyntaxOff
\let\ctexrevnum\ExplFileVersion
\expandafter\let\csname ver@thumbpdf.sty\endcsname\fmtversion
\documentclass[numbered,full,a4paper]{l3doc}
\usepackage{amsmath}
\usepackage{xeCJK}
\usepackage{zhnumber}
\usepackage{fvrb-ex}
\usepackage{indentfirst}
\usepackage{geometry}
\geometry{includemp,hmargin={0mm,15mm},vmargin=15mm,footskip=7mm}
\BeforeBeginEnvironment{SideBySideExample}{\vskip1ex\relax}
\hypersetup{pdfstartview=FitH}
\setlist{noitemsep,topsep=\smallskipamount}
\linespread{1.1}
\setmainfont{TeX Gyre Pagella}
\setsansfont{CMU Sans Serif}
\setmonofont[
  UprightFont=* Light, BoldFont=* Bold,
  SlantedFont=* Light Oblique]{CMU Typewriter Text}
\xeCJKDeclareCharClass{CJK}{ "25CB }
\setCJKmainfont[BoldFont=Adobe Heiti Std,ItalicFont=Adobe Kaiti Std]{Adobe Song Std}
\setCJKmonofont{Adobe Kaiti Std}
\xeCJKsetup{PunctStyle=kaiming}
\def\ctexkitrev#1{%
  \href{http://code.google.com/p/ctex-kit/source/detail?r=#1}{\texttt{ctex-kit} rev#1}}
\makeatletter
\def\LaTeX{\hologo{LaTeX}}
\def\pdfTeX{\hologo{pdfTeX}}
\def\pdfLaTeX{\hologo{pdfLaTeX}}
\def\LuaLaTeX{\hologo{LuaLaTeX}}
\def\XeLaTeX{\hologo{XeLaTeX}}
\def\TF{true\orvar{}false}
\def\TTF{\defaultvar{true}\orvar false}
\def\TFF{true\orvar\defaultvar{false}}
\def\orvar{\textup{\textbar}}
\def\defaultvar{\textbf}
\def\argbrace#1{\{#1\}}
\preto\MacroFont{\linespread{1}}
\appto\MacroFont{\hyphenchar\font\m@ne}
\preto\AltMacroFont{\linespread{1}}
\appto\AltMacroFont{\hyphenchar\font\m@ne}
\ExplSyntaxOn
\cs_set_protected:Npn \__codedoc_special_index_aux:nnnnn #1#2#3#4#5
  {
    \HD@target
    \__codedoc_special_index_set:Nn \l__codedoc_index_escaped_macro_tl {#2}
    \str_if_eq:onTF { \@currenvir } { macrocode }
      { \codeline@wrindex }
      { \index }
      {
        \tl_if_empty:nF { #3 #4 }
          { #3 \actualchar #4 \levelchar }
        #1
        \actualchar
        {
          \token_to_str:N \verbatim@font \c_space_tl
          \l__codedoc_index_escaped_macro_tl
        }
        \encapchar
        hdclindex{\the\c@HD@hypercount}{#5}
      }
  }
\ExplSyntaxOff
\makeatother
\def\indexname{代码索引}
\IndexPrologue{%
  \section*{\indexname}
  \markboth{\indexname}{\indexname}
  斜体的数字表示对应项说明所在的页码,下划线的数字表示定义所在的代码行号,而直立体的
  数字表示对应项使用时所在的行号。}
\begin{document}
  \DocInput{\jobname.dtx}
  \newgeometry{margin=15mm,footskip=7mm}
  \PrintIndex
\end{document}
%</driver>
% \fi
%
% \CheckSum{1043}
% \GetFileInfo{\jobname.sty}
%
% \title{\bfseries\pkg{zhnumber} 宏包}
% \author{李清\\ \path{sobenlee@gmail.com}}
% \date{\filedate\qquad\fileversion\thanks{\ctexkitrev{\ctexrevnum}.}}
% \maketitle
%
% \begin{documentation}
%
% \section{简介}
% \pkg{zhnumber} 宏包用于将阿拉伯数字按照中文格式输出。相比于 \pkg{CJKnumb},它提供
% 的三个格式转换命令 \tn{zhnumber},\tn{zhdigits} 和 \tn{zhnum} 都是可以适当展开的,
% 可以正常使用于 |PDF| 书签和交叉引用。
%
% \pkg{zhnumber} 支持 |GBK|,|Big5| 和 |UTF8| 编码,依赖 \hologo{LaTeX3} 项目的
% \pkg{expl3},\pkg{xparse} 和 \pkg{l3keys2e} 宏包。
%
% \section{使用方法}
%
% \begin{function}[updated=2012-5-25]{encoding}
%   \begin{syntax}
%     encoding = \meta{GBK\orvar{}Big5\orvar{}UTF8}
%   \end{syntax}
%   用于指定编码的宏包选项,可以在调用宏包的时候设定,也可以用 \tn{zhnumsetup} 在导言区内设定。
%   对于 \XeLaTeX 和 \LuaLaTeX,不用指定编码,宏包将自动使用 |UTF8| 编码。只有 \LaTeX
%   和 pdf\LaTeX 需要指定编码,如果没有指定,默认将使用 |GBK|,并且此时 \pkg{zhnumber}
%   宏包应该在 \pkg{CJK} 或 \pkg{CJKutf8} 宏包之后载入。
% \end{function}
%
% \begin{function}{\zhnumber}
%   \begin{syntax}
%     \tn{zhnumber} \Arg{number}
%   \end{syntax}
%   以中文格式输出数字。这里的数字可以是整数、小数和分数。例如
%   \begin{SideBySideExample}[frame=single,numbers=left,xrightmargin=.6\linewidth,gobble=5]
%     \zhnumber{2012020120}\\
%     \zhnumber{2 012 020 120}\\
%     \zhnumber{2,012,020,120}\\
%     \zhnumber{2012.020120}\\
%     \zhnumber{2012.}\\
%     \zhnumber{.2012}\\
%     \zhnumber{20120/20120}\\
%     \zhnumber{/2012}\\
%     \zhnumber{2012/}\\
%     \zhnumber{201;2020/120}
%   \end{SideBySideExample}
% \end{function}
%
% \begin{function}{\zhdigits}
%   \begin{syntax}
%     \tn{zhdigits}   \Arg{number}
%     \tn{zhdigits} * \Arg{number}
%   \end{syntax}
%   将阿拉伯数字转换为中文数字串。缺省状态下,\tn{zhdigits} 将 0 映射为〇,如果需要
%   将其映射为零,可以使用带星号的形式。例如
%   \begin{SideBySideExample}[frame=single,numbers=left,xrightmargin=.6\linewidth,gobble=5]
%     \zhdigits{2012020120}\\
%     \zhdigits*{2012020120}
%   \end{SideBySideExample}
% \end{function}
%
% \begin{function}{\zhnum}
%   \begin{syntax}
%     \tn{zhnum} \Arg{counter}
%   \end{syntax}
%   与 |\roman| 等类似,用于将 \LaTeX 计数器的值转换为中文数字。例如
%   \begin{SideBySideExample}[frame=single,numbers=left,xrightmargin=.6\linewidth,gobble=5]
%     \zhnum{section}
%   \end{SideBySideExample}
% \end{function}
%
% \begin{function}[added=2012-5-25]{\zhweekday}
%   \begin{syntax}
%     \tn{zhweekday} \Arg{yyyy/mm/dd}
%   \end{syntax}
%   输出日期当天的星期。例如
%   \begin{SideBySideExample}[frame=single,numbers=left,xrightmargin=.6\linewidth,gobble=5]
%     \zhweekday{2012/5/20}
%   \end{SideBySideExample}
% \end{function}
%
% \begin{function}[added=2012-5-25]{\zhdate}
%   \begin{syntax}
%     \tn{zhdate}   \Arg{yyyy/mm/dd}
%     \tn{zhdate} * \Arg{yyyy/mm/dd}
%   \end{syntax}
%   以中文格式输出日期,其中带 |*| 的命令还输出星期。例如
%   \begin{SideBySideExample}[frame=single,numbers=left,xrightmargin=.6\linewidth,gobble=5]
%     \zhdate{2012/5/21}\\
%     \zhdate*{2012/5/21}
%   \end{SideBySideExample}
% \end{function}
%
% \begin{function}[added=2012-5-25]{\zhtoday}
%   与 |\today| 类似,以中文输出当天的日期。例如
%   \begin{SideBySideExample}[frame=single,numbers=left,xrightmargin=.6\linewidth,gobble=5]
%     \zhtoday
%   \end{SideBySideExample}
% \end{function}
%
% \begin{function}[added=2012-5-25]{\zhtime}
%   \begin{syntax}
%     \tn{zhtime} \Arg{hh:mm}
%   \end{syntax}
%   以中文格式输出时间。例如
%   \begin{SideBySideExample}[frame=single,numbers=left,xrightmargin=.6\linewidth,gobble=5]
%     \zhtime{23:56}
%   \end{SideBySideExample}
% \end{function}
%
% \begin{function}[added=2012-5-25]{\zhcurrtime}
%   输出当前的时间。例如
%   \begin{SideBySideExample}[frame=single,numbers=left,xrightmargin=.6\linewidth,gobble=5]
%     \zhcurrtime
%   \end{SideBySideExample}
% \end{function}
%
% \begin{function}[added=2012-5-25]{\zhnumExtendScaleMap}
%   \begin{syntax}
%     \tn{zhnumExtendScaleMap} \oarg{character} \argbrace{\meta{character_1}, \meta{character_2}, ..., \meta{character_n}}
%   \end{syntax}
%   缺省状态下 \tn{zhnumber} 能正确中文格式化的最大整数是 $10^{48}-1$,\tn{zhdigits} 不受
%   这个大小的限制。可以通过 \tn{zhnumExtendScaleMap} 来扩展 \tn{zhnumber}。
%   \meta{character_i} 设置 $10^{4(i+11)}$。若给出可选项 \meta{character},则当
%   数字大于 $10^{4(n+12)}-1$ 时,统一用 \meta{character} 设置输出数字的进位。
% \end{function}
%
% \begin{function}{\zhnumsetup}
%   \begin{syntax}
%     \tn{zhnumsetup} \argbrace{\meta{key_1}=\meta{val_1}, \meta{key_2}=\meta{val_2}, ...}
%   \end{syntax}
%   用于在导言区或文档中,设置中文数字的输出格式。目前可以设置的 \meta{key} 如下介绍。
% \end{function}
%
% \begin{function}[added=2012-5-25]{time}
%   \begin{syntax}
%     time =  \argbrace{\meta{Arabic}\orvar\meta{Chinese}}
%   \end{syntax}
%   设置日期和时间的数字格式,\meta{Arabic} 为阿拉伯数字,而 \meta{Chinese} 为中文数字。
%   默认使用阿拉伯数字。例如
%   \begin{SideBySideExample}[frame=single,numbers=left,xrightmargin=.6\linewidth,gobble=5]
%     \zhnumsetup{time=Chinese}
%     \zhtoday\zhcurrtime
%   \end{SideBySideExample}
% \end{function}
%
% \begin{function}[updated=2012-5-25]{style}
%   \begin{syntax}
%     style = \argbrace{\meta{Simplified}\orvar\meta{Traditional}\orvar\meta{Normal}\orvar\meta{Financial}\orvar\meta{Ancient}}
%   \end{syntax}
%   意义分别为
%
%   \begin{description}[font=\mdseries\ttfamily,align=right,labelsep=1em,labelindent=-1em,leftmargin=*]
%     \item[Simplified]  以简体中文输出数字(对 |Big5| 编码无效);
%     \item[Traditional] 以繁体中文输出数字(对 |Big5| 编码无效);
%     \item[Normal] 以小写形式输出中文数字;
%     \item[Financial]  以大写形式输出中文数字;
%     \item[Ancient] 以廿输出 20,以卅输出 30,以卌输出 40,以皕输出 200。
%   \end{description}
%
%   可以设置 |style| 为其中一个,也可以是前三个与后两个的适当组合,默认是简体小写。例如
%   \begin{SideBySideExample}[frame=single,numbers=left,xrightmargin=.4\linewidth,gobble=5]
%     \zhnumsetup{style={Traditional,Financial}}
%     \zhnumber{62012.3}\\
%     \zhnumsetup{style=Ancient}
%     \zhnumber{21}
%   \end{SideBySideExample}
% \end{function}
%
% \begin{function}{null}
%   \begin{syntax}
%     null = \meta{\TFF}
%   \end{syntax}
%   缺省状态下,除了 \tn{zhdigits} 外,其它的格式转换命令,将 0 映射成零,如果需要将 0 映射
%   成〇,可以使用这个选项。\strut
% \end{function}
%
% \pkg{zhnumber} 提供下列选项来控制阿拉伯数字的中文映射。
% \begin{verbatim}[frame=single]
%   - -0 0 1 2 3 4 5 6 7 8 9 10 20 30 40 200
%   E2 E3 E4 E8 E12 E16 E20 E24 E28 E32 E36 E40 E44
%   F0 F1 F2 F3 F4 F5 F6 F7 F8 F9 F10 FE2 FE3
%   dot and parts
%   year month day hour minute weekday mon tue wed thu fri sat sun
% \end{verbatim}
% 其中 |-| 设置负,|-0| 设置〇,|dot| 设置小数的点,|and| 和 |parts| 分别设置分数
% 的“又”和“分之”,|E|$n$ 设置 $10^n$,而 |F|$n$ 设置数字 $n$ 的大写。其它的选项同
% 字面意思,不再赘述。例如
% \begin{verbatim}[frame=single]
%   \zhnumsetup{2={两}}
% \end{verbatim}
% 可以将 2 映射成两。需要说明的是,\pkg{zhnumber} 将优先使用这里的设置,所以可能会影响
% 到 |style| 选项。如果要恢复 |style| 的功能,可以使用 |reset| 选项。
%
% \begin{function}[updated=2012-5-25]{reset}
%   \begin{syntax}
%     reset
%   \end{syntax}
%   用于恢复 \pkg{zhnumber} 对阿拉伯数字的初始化映射。\pkg{zhnumber} 的中文数字初始化
%   设置见源代码(第 \ref{sec:zhnum-map} 节)。
% \end{function}
%
% \begin{function}{\zhnumber,\zhdigits,\zhnum}
%   \begin{syntax}
%     \tn{zhnumber}   \oarg{options} \Arg{number}
%     \tn{zhdigits} * \oarg{options} \Arg{number}
%     \tn{zhnum}      \oarg{options} \Arg{counter}
%   \end{syntax}
%   如果只改变当前数字的中文输出格式,可以使用带选项的格式转换命令,其中 \meta{options}
%   与 \tn{zhnumsetup} 的参数相同,如上所介绍。这些带了选项的命令是不可展开的,在某些场合使
%   用时要小心。
% \end{function}
%
% \end{documentation}
%
% \StopEventually{}
%
% \begin{implementation}
%
% \section{\pkg{zhnumber} 宏包代码实现}
%
%    \begin{macrocode}
%<*package>
%    \end{macrocode}
%
%    \begin{macrocode}
%<@@=zhnum>
%    \end{macrocode}
%
%    \begin{macrocode}
\msg_new:nnn { zhnumber } { l3-too-old }
  {
    Support~package~'expl3'~too~old. \\\\
    Please~update~an~up~to~date~version~of~the~bundles\\\\
    'l3kernel'~and~'l3packages'\\\\
    using~your~TeX~package~manager~or~from~CTAN.
  }
\@ifpackagelater { expl3 } { 2014/08/25 } { }
  { \msg_error:nn  { zhnumber }  { l3-too-old } }
%    \end{macrocode}
%
%    \begin{macrocode}
\RequirePackage { xparse , l3keys2e }
%    \end{macrocode}
%
% \begin{macro}{\zhnumber}
% 用于将输入的数字按照中文格式输出。
%    \begin{macrocode}
\DeclareExpandableDocumentCommand \zhnumber { +o +m }
  {
    \IfNoValueTF {#1}
      { \zhnum_number:n {#2} }
      { \zhnumberwithoptions {#1} {#2} }
  }
%    \end{macrocode}
% \end{macro}
%
% \begin{macro}{\zhnumberwithoptions}
% 带选项的用户函数。
%    \begin{macrocode}
 \NewDocumentCommand \zhnumberwithoptions { +m +m }
  {
    \group_begin:
      \keys_set:nn { zhnum / options } {#1}
      \zhnum_number:n {#2}
    \group_end:
  }
%    \end{macrocode}
% \end{macro}
%
% \begin{macro}[internal]{\zhnum_number:n}
% \begin{macro}[aux]{\@@_number:www}
% 先判断输入的是小数还是分数。
%    \begin{macrocode}
\cs_new:Npn \zhnum_number:n #1
  { \@@_number:www #1 . \q_nil . \q_stop }
\cs_new:Npn \@@_number:www #1 . #2 . #3 \q_stop
  {
    \quark_if_nil:nTF {#2}
      { \@@_integer_or_fraction:www #1 / \q_nil / \q_stop }
      { \zhnum_decimal:nn {#1} {#2} }
  }
%    \end{macrocode}
% \end{macro}
% \end{macro}
%
% \begin{macro}[aux]{\@@_integer_or_fraction:www}
% 判断是否输入的是分数。
%    \begin{macrocode}
\cs_new:Npn \@@_integer_or_fraction:www #1 / #2 / #3 \q_stop
  {
    \quark_if_nil:nTF {#2}
      { \zhnum_integer:n {#1} }
      { \@@_fraction:wwww #2 \q_mark #1 ; \q_nil ; \q_stop }
  }
%    \end{macrocode}
% \end{macro}
%
% \begin{macro}[aux]{\@@_fraction:wwww}
% 对分数进行预处理。
%    \begin{macrocode}
\cs_new:Npn \@@_fraction:wwww #1 \q_mark #2 ; #3 ; #4 \q_stop
  {
    \quark_if_nil:nTF {#3}
      {
        \zhnum_blank_to_zero:f {#1}
        \c_@@_parts_tl
        \zhnum_blank_to_zero:f {#2}
      }
      {
        \tl_if_blank:fF {#2}
          {
            \zhnum_number:n {#2}
            \c_@@_and_tl
          }
        \zhnum_blank_to_zero:f {#1}
        \c_@@_parts_tl
        \zhnum_blank_to_zero:f {#3}
      }
  }
\cs_generate_variant:Nn \tl_if_blank:nF { f }
%    \end{macrocode}
% \end{macro}
%
% \begin{macro}[internal]{\zhnum_decimal:nn}
% 对小数进行预处理。
%    \begin{macrocode}
\cs_new:Npn \zhnum_decimal:nn #1#2
  {
    \zhnum_blank_to_zero:f {#1} \c_@@_dot_tl
    \tl_if_blank:fTF {#2}
      { \c_@@_zero_tl }
      { \zhnum_digits_zero:n {#2} }
  }
\cs_generate_variant:Nn \tl_if_blank:nTF { f }
%    \end{macrocode}
% \end{macro}
%
% \begin{macro}[internal]{\zhnum_blank_to_zero:n}
% 输出小数的整数位。
%    \begin{macrocode}
\cs_new:Npn \zhnum_blank_to_zero:n #1
  {
    \tl_if_blank:nTF {#1}
      { \c_@@_zero_tl }
      { \zhnum_number:n {#1} }
  }
\cs_generate_variant:Nn \zhnum_blank_to_zero:n { f }
%    \end{macrocode}
% \end{macro}
%
% \begin{macro}{\zhnum,\zhnumberwithoptions}
% 用于将 \LaTeX{} 计数器按中文格式输出。
%    \begin{macrocode}
\DeclareExpandableDocumentCommand \zhnum { o m }
  {
    \IfNoValueTF {#1}
      { \zhnum_counter:n {#2} }
      { \zhnumwithoptions {#1} {#2} }
  }
\NewDocumentCommand \zhnumwithoptions { m m }
  {
    \group_begin:
      \keys_set:nn { zhnum / options } {#1}
      \zhnum_counter:n {#2}
    \group_end:
  }
%    \end{macrocode}
% \end{macro}
%
% \begin{macro}[internal]{\zhnum_counter:n}
% 可以直接通过比较 \LaTeX{} 计数器的值来得到符号和绝对值。
%    \begin{macrocode}
\cs_new_nopar:Npn \zhnum_counter:n #1
  { \exp_args:Nc \zhnum_counter:Nn  { c@#1 } {#1} }
\cs_new:Npn \zhnum_counter:Nn #1#2
  {
    \token_if_int_register:NTF #1
      { \zhnum_int:n {#1} }
      { \__msg_expandable_error:n { `#2' is not a LaTeX counter. } }
  }
\cs_new:Npn \zhnum_int:n #1
  {
    \int_compare:nNnTF {#1} > \c_zero
      { \zhnum_parse_number:f { \int_eval:n {#1} } }
      {
        \int_compare:nNnTF {#1} < \c_zero
          {
            \c_@@_minus_tl
            \zhnum_parse_number:f { \int_eval:n { - #1 } }
          }
          { \c_@@_zero_tl }
      }
  }
%    \end{macrocode}
% \end{macro}
%
% \begin{macro}[internal]{\zhnum_integer:n}
% 对整数的处理。这个函数抄录自 \pkg{l3bigint} 的 \cs{__bingint_read_do:nn}。它可以
% 正确取得符号,去掉多余的零,还可以循环展开数字。但它在遇到非数字的时候就停止了
% 循环,我们可能需要非数字(例如逗号)来作为分隔符号。因此对它略作修改,跳过非数字。
%    \begin{macrocode}
\cs_new:Npn \zhnum_integer:n #1
  {
    \exp_after:wN \@@_read_integer:ww
    \tex_number:D
      \exp_after:wN \@@_read_sign_loop:N
      \tex_romannumeral:D -`0 \use:n
      #1 \s__stop ;
  }
\cs_new:Npn \@@_read_sign_loop:N #1
  {
    \if:w + \if:w - \exp_not:N #1 + \fi: \exp_not:N #1
      \exp_after:wN \@@_read_sign_loop:N
      \tex_romannumeral:D -`0 \exp_after:wN \use:n
    \else:
      1 \exp_after:wN ;
      \tex_romannumeral:D -`0
        \exp_after:wN \@@_read_zeros_loop:N
        \exp_after:wN #1
    \fi:
  }
\cs_new:Npn \@@_read_zeros_loop:N #1
  {
    \if:w 0 \exp_not:N #1
      \exp_after:wN \@@_read_zeros_loop:N
      \tex_romannumeral:D -`0 \exp_after:wN \use:n
    \else:
      \exp_after:wN \@@_read_abs_loop:N
      \exp_after:wN #1
    \fi:
  }
\cs_new:Npn \@@_read_abs_loop:N #1
  {
    \if_int_compare:w \c_nine < 1 \exp_not:N #1 \exp_stop_f:
      \exp_after:wN #1
      \tex_romannumeral:D -`0
        \exp_after:wN \@@_read_abs_loop:N
        \tex_romannumeral:D -`0 \exp_after:wN \use:n
    \else:
      \str_if_eq:nnTF {#1} { \s__stop }
        { \exp_after:wN \exp_stop_f: }
        {
          \exp_after:wN \@@_read_abs_loop:N
          \tex_romannumeral:D -`0 \exp_after:wN \use:n
        }
    \fi:
  }
\cs_new:Npn \@@_read_integer:ww #1 ; #2 ;
  {
    \tl_if_empty:nTF {#2}
      { \c_@@_zero_tl }
      {
        \int_compare:nNnF {#1} = \c_one
          { \c_@@_minus_tl }
        \zhnum_parse_number:n {#2}
      }
  }
%    \end{macrocode}
% \end{macro}
%
% \begin{macro}[internal]{\zhnum_parse_number:n}
%    \begin{macrocode}
\cs_new_nopar:Npn \zhnum_parse_number:n #1
  { \exp_args:Nf \@@_parse_number:nn { \tl_count:n {#1} } {#1} }
\cs_new_nopar:Npn \@@_parse_number:nn #1#2
  { \exp_args:Nf \@@_parse_number:nnn { \int_mod:nn {#1} \c_four } {#1} {#2} }
\cs_new_nopar:Npn \@@_parse_number:nnn #1#2#3
  {
    \int_compare:nNnTF {#2} < \c_two
      { \zhnum_digit_map:n {#3} }
      {
        \zhnum_split_number:fNNfn { \zhnum_insert_zeros:n {#1} #3 }
          \c_true_bool \c_true_bool
          {
            \int_compare:nNnTF {#1} = \c_zero
              { \int_eval:n { (#2) / \c_four - \c_one } }
              { \int_div_truncate:nn {#2} \c_four }
          }
          { \c_zero }
      }
  }
\cs_generate_variant:Nn \zhnum_parse_number:n { f }
%    \end{macrocode}
% \end{macro}
%
% \begin{macro}[internal]{\zhnum_insert_zeros:n}
% 为了处理的方便,在整数前面补上适当的 $0$,使其位数可以被 $4$ 整除。
%    \begin{macrocode}
\cs_new_nopar:Npn \zhnum_insert_zeros:n #1
  {
    \if_case:w \etex_numexpr:D #1 - \c_one \scan_stop:
      \or: \exp_after:wN \use_none:n
      \or: \exp_after:wN \use_none:nn
    \else: \exp_after:wN \use_none:nnn
    \fi:
    0 0 0
  }
%    \end{macrocode}
% \end{macro}
%
% \begin{macro}[internal]{\zhnum_split_number:nNNnn}
% 将输入的整数由高位到低位,以四位为一段进行处理。
%    \begin{macrocode}
\cs_new_nopar:Npn \zhnum_split_number:nNNnn #1#2#3#4#5
  {
    \@@_split_number:fnNNnn { \zhnum_number_item:nn {#1} {#5} }
      {#1} #2#3 {#4} {#5}
  }
\cs_new_nopar:Npn \@@_split_number:nnNNnn #1#2#3#4#5#6
  {
    \int_compare:nNnTF {#1} = \c_zero
      { \@@_split_number_aux:NNfnnn \c_false_bool \c_true_bool }
      {
        \bool_if:NF #3 { \c_@@_zero_tl }
        \zhnum_process_number:NNn #3#4 {#1}
        \zhnum_scale_map:n { #5 - #6 }
        \int_compare:nNnTF { \int_mod:nn {#1} \c_ten } = \c_zero
          { \@@_split_number_aux:NNfnnn \c_false_bool \c_true_bool }
          { \@@_split_number_aux:NNfnnn \c_true_bool \c_false_bool }
      }
    { \int_eval:n { #6 + \c_one } } {#2} {#5} {#6}
  }
\cs_new_nopar:Npn \@@_split_number_aux:NNnnnn #1#2#3#4#5#6
  {
    \int_compare:nNnF {#5} = {#6}
      { \zhnum_split_number:nNNnn {#4} #1#2 {#5} {#3} }
  }
\cs_generate_variant:Nn \zhnum_split_number:nNNnn { fNNf }
\cs_generate_variant:Nn \@@_split_number:nnNNnn { f }
\cs_generate_variant:Nn \@@_split_number_aux:NNnnnn { NNf }
%    \end{macrocode}
% \end{macro}
%
% \begin{macro}[internal]{\zhnum_number_item:nn}
% 截取整数的其中四位数。
%    \begin{macrocode}
\cs_new_nopar:Npn \zhnum_number_item:nn #1#2
  { \@@_number_item:nNNNN {#2} #1 \q_recursion_stop }
\cs_new_nopar:Npn \@@_number_item:nNNNN #1#2#3#4#5
  {
    \int_compare:nNnTF {#1} = \c_zero
      { \@@_recursion_stop:NNNNw #2#3#4#5 }
      { \@@_number_item:fNNNN { \int_eval:n { #1 - \c_one } } }
  }
\cs_generate_variant:Nn \@@_number_item:nNNNN { f }
\cs_new_nopar:Npn \@@_recursion_stop:NNNNw #1#2#3#4#5 \q_recursion_stop
  { #1#2#3#4 }
%    \end{macrocode}
% \end{macro}
%
% \begin{macro}[internal]{\zhnum_process_number:NNn,\zhnum_process_number:NNNNNN}
% 对四位数字按情况进行处理。
%    \begin{macrocode}
\cs_new_nopar:Npn \zhnum_process_number:NNn #1#2#3
  { \zhnum_process_number:NNNNNN #3#1#2 }
\cs_new_nopar:Npn \zhnum_process_number:NNNNNN #1#2#3#4#5#6
  {
    \int_compare:nNnTF {#1} = \c_zero
      { \bool_if:NF #6 { \c_@@_zero_tl } }
      { \zhnum_digit_map:n {#1} \c_@@_thousand_tl }
    \int_compare:nNnTF {#2} = \c_zero
      { \int_compare:nNnT { #1 * (#3#4) } > \c_zero { \c_@@_zero_tl } }
      {
        \bool_if:nTF
          { \l_@@_ancient_bool && \int_compare_p:nNn {#2} = \c_two }
          { \zhnum_digit_map:n { #2 00 } }
          { \zhnum_digit_map:n {#2} \c_@@_hundred_tl }
      }
    \int_compare:nNnTF {#3} = \c_zero
      { \int_compare:nNnT { #2 * #4 } > \c_zero { \c_@@_zero_tl } }
      {
        \bool_if:nF
          {
            \int_compare_p:nNn {#3}   = \c_one  &&
            \int_compare_p:nNn {#1#2} = \c_zero && #6 && #5
          }
          {
            \bool_if:nTF
              {
                \l_@@_ancient_bool                   &&
                ( \int_compare_p:nNn {#3} = \c_two   ||
                  \int_compare_p:nNn {#3} = \c_three ||
                  \int_compare_p:nNn {#3} = \c_four )
              }
              { \zhnum_digit_map:n { #3 0 } \use_none:n }
              { \zhnum_digit_map:n {#3} }
          }
        \c_@@_ten_tl
      }
    \int_compare:nNnF {#4} = \c_zero { \zhnum_digit_map:n {#4} }
  }
%    \end{macrocode}
% \end{macro}
%
% \begin{macro}{\zhdigits,\zhdigitswithoptions}
% 将输入的数字输出为中文数字串输出。
%    \begin{macrocode}
\DeclareExpandableDocumentCommand \zhdigits { s o m }
  {
    \IfNoValueTF {#2}
      { \zhnum_digits:Nn #1 {#3} }
      {
        \IfBooleanTF #1
          { \zhdigitswithoptions * }
          { \zhdigitswithoptions }
        {#2} {#3}
      }
  }
\NewDocumentCommand \zhdigitswithoptions { s m m }
  {
    \group_begin:
      \keys_set:nn { zhnum / options } {#2}
      \zhnum_digits:Nn #1 {#3}
    \group_end:
  }
%    \end{macrocode}
% \end{macro}
%
% \begin{macro}[internal]{\zhnum_digits_zero:n,\zhnum_digits_null:n}
% 快捷方式。
%    \begin{macrocode}
\cs_new_nopar:Npn \zhnum_digits_zero:n
  { \zhnum_digits:Nn \BooleanTrue }
\cs_new_nopar:Npn \zhnum_digits_null:n
  { \zhnum_digits:Nn \BooleanFalse }
\cs_generate_variant:Nn \zhnum_digits_null:n { V }
%    \end{macrocode}
% \end{macro}
%
% \begin{macro}[internal]{\zhnum_digits:Nn}
% 与 \cs{zhnum_integer:n} 类似,但不用去掉多余的零。
%    \begin{macrocode}
\cs_new:Npn \zhnum_digits:Nn #1#2
  {
    \exp_after:wN \@@_read_digits:Nww \exp_after:wN #1
    \tex_number:D
      \exp_after:wN \@@_read_sign_loop:NN \exp_after:wN #1
      \tex_romannumeral:D -`0 \use:n
      #2 \s__stop ;
  }
\cs_new:Npn \@@_read_sign_loop:NN #1#2
  {
    \if:w + \if:w - \exp_not:N #2 + \fi: \exp_not:N #2
      \exp_after:wN \@@_read_sign_loop:NN \exp_after:wN #1
      \tex_romannumeral:D -`0 \exp_after:wN \use:n
    \else:
      1 \exp_after:wN ;
      \tex_romannumeral:D -`0
        \exp_after:wN \@@_read_digits_loop:NN
        \exp_after:wN #1
        \exp_after:wN #2
    \fi:
  }
\cs_new:Npn \@@_read_digits_loop:NN #1#2
  {
    \if_int_compare:w \c_nine < 1 \exp_not:N #2 \exp_stop_f:
      \exp_after:wN \@@_output_digits:NN \exp_after:wN #1 \exp_after:wN #2
      \tex_romannumeral:D -`0
        \exp_after:wN \@@_read_digits_loop:NN \exp_after:wN #1
        \tex_romannumeral:D -`0 \exp_after:wN \use:n
    \else:
      \str_case:nnF {#2}
        {
          { \s__stop }
            { \exp_after:wN \exp_stop_f: }
          { . }
            {
              \exp_after:wN \exp_stop_f: \exp_after:wN \c_@@_dot_tl
              \tex_romannumeral:D -`0
                \exp_after:wN \@@_read_digits_loop:NN \exp_after:wN #1
                \tex_romannumeral:D -`0 \exp_after:wN \use:n
            }
        }
        {
          \exp_after:wN \@@_read_digits_loop:NN \exp_after:wN #1
          \tex_romannumeral:D -`0 \exp_after:wN \use:n
        }
    \fi:
  }
\cs_new:Npn \@@_read_digits:Nww #1 #2 ; #3 ;
  {
    \tl_if_empty:nTF {#3}
      { \IfBooleanTF #1 { \c_@@_zero_tl } { \c_@@_null_tl } }
      {
        \int_compare:nNnF {#2} = \c_one
          { \c_@@_minus_tl }
        #3
      }
  }
\cs_new:Npn \@@_output_digits:NN #1#2
  {
    \exp_after:wN \exp_stop_f:
      \cs:w
        c_@@_
          \if_int_compare:w #2 = \c_zero
            \IfBooleanTF #1 { zero } { null }
          \else:
            \zhnum_int_to_word:n {#2}
          \fi:
        _tl
      \cs_end:
  }
%    \end{macrocode}
% \end{macro}
%
% \begin{macro}{\zhdate}
% 输出中文日期。
%    \begin{macrocode}
\DeclareExpandableDocumentCommand \zhdate { s m }
  {
    \@@_date:www #2 \q_stop
    \IfBooleanT {#1}
      {  \@@_week_day:www #2 \q_stop }
  }
\cs_new_nopar:Npn \@@_date:www #1/#2/#3 \q_stop
  {
    \zhnum_check_time:Nn \zhnum_digits_null:n {#1} \c_@@_year_tl
    \zhnum_check_time:Nn \zhnum_integer:n {#2} \c_@@_month_tl
    \zhnum_check_time:Nn \zhnum_integer:n {#3} \c_@@_day_tl
  }
%    \end{macrocode}
% \end{macro}
%
% \begin{macro}{\zhtoday}
% 输出当天日期。
%    \begin{macrocode}
\cs_new_nopar:Npn \zhtoday
  {
    \zhnum_check_time:Nn \zhnum_digits_null:n \tex_year:D \c_@@_year_tl
    \zhnum_check_time:Nn \zhnum_int:n \tex_month:D \c_@@_month_tl
    \zhnum_check_time:Nn \zhnum_int:n \tex_day:D   \c_@@_day_tl
  }
%    \end{macrocode}
% \end{macro}
%
% \begin{macro}[internal]{\zhnum_check_time:Nn}
% 判断是用中文数字还是用阿拉伯数组。
%    \begin{macrocode}
\cs_new_nopar:Npn \zhnum_check_time:Nn #1
  { \bool_if:NTF \l_@@_time_bool {#1} { \int_eval:n } }
%    \end{macrocode}
% \end{macro}
%
% \begin{macro}{\zhweekday}
% 输出星期
%    \begin{macrocode}
\cs_new_nopar:Npn \zhweekday #1
  { \@@_week_day:www #1 \q_stop }
%    \end{macrocode}
% \end{macro}
%
% \begin{macro}[internal]{\@@_week_day:www}
% 用 Zeller 公式计算的结果 $h$ 与实际星期的关系是 $d=h+5\pmod7+1$。
%    \begin{macrocode}
\cs_new_nopar:Npn \@@_week_day:www #1/#2/#3 \q_stop
  {
    \if_case:w \etex_numexpr:D \zhnum_Zeller:nnn {#1} {#2} {#3} \scan_stop:
           \c_@@_sat_tl
      \or: \c_@@_sun_tl
      \or: \c_@@_mon_tl
      \or: \c_@@_tue_tl
      \or: \c_@@_wed_tl
      \or: \c_@@_thu_tl
      \or: \c_@@_fri_tl
    \fi:
  }
%    \end{macrocode}
% \end{macro}
%
% \begin{macro}[internal]{\zhnum_Zeller:nnn,\zhnum_Zeller_aux:Nnnn,\zhnum_two_digits:n}
% 用 Zeller 公式\footnote{\url{http://en.wikipedia.org/wiki/Zeller's_congruence}}
% 计算星期几。
%    \begin{macrocode}
\cs_new_nopar:Npn \zhnum_Zeller:nnn #1#2#3
  {
    \int_compare:nNnTF
      { #1 \zhnum_two_digits:n {#2} \zhnum_two_digits:n {#3} } > { 1582 10 04 }
      { \@@_Zeller_aux:Nnnn \zhnum_Zeller_Gregorian:nnn }
      { \@@_Zeller_aux:Nnnn \zhnum_Zeller_Julian:nnn }
    {#1} {#2} {#3}
  }
\cs_new_nopar:Npn \@@_Zeller_aux:Nnnn  #1#2#3#4
  {
    \int_compare:nNnTF {#3} < \c_three
      { #1 { #2 - \c_one } { #3 + \c_twelve } {#4} }
      { #1 {#2} {#3} {#4} }
  }
\cs_new_nopar:Npn \zhnum_two_digits:n #1
  {
    \int_compare:nNnT {#1} < \c_ten { 0 }
    \int_eval:n {#1}
  }
%    \end{macrocode}
% \end{macro}
%
% \begin{macro}[internal]{\zhnum_Zeller_Gregorian:nnn}
% 格里历(\zhdate{1582/10/15}及以后)的计算公式
% \[
%   h = \biggl(q + \biggl\lfloor\frac{26(m+1)}{10}\biggr\rfloor + Y +
%   \biggl\lfloor\frac Y4\biggr\rfloor + 6\biggl\lfloor\frac Y{100}\biggr\rfloor
%   + \biggl\lfloor\frac Y{400}\biggr\rfloor\biggr) \pmod 7
% \]
% 其中 $Y$ 为年,$m$ 为月,$q$ 为日;若 $m=1,2$,则令 $m\mathbin{{+}{=}}12$,同时 $Y\mathop{--}{}$。
%    \begin{macrocode}
\cs_new_nopar:Npn \zhnum_Zeller_Gregorian:nnn #1#2#3
  {
    \int_mod:nn
      {
          (#3)
        + \int_div_truncate:nn { 26 * ( #2 + \c_one ) } \c_ten
        + (#1)
        + \int_div_truncate:nn {#1} \c_four
        + \c_six * \int_div_truncate:nn {#1} \c_one_hundred
        + \int_div_truncate:nn {#1} { 400 }
      }
      { \c_seven }
  }
%    \end{macrocode}
% \end{macro}
%
% \begin{macro}[internal]{\zhnum_Zeller_Julian:nnn}
% 儒略历(\zhdate{1582/10/04}及以前)的计算公式
% \[
%   h = \biggl(q + \biggl\lfloor\frac{26(m+1)}{10}\biggr\rfloor + Y +
%   \biggl\lfloor\frac Y4\biggr\rfloor + 5\biggr) \pmod 7
% \]
%    \begin{macrocode}
\cs_new_nopar:Npn \zhnum_Zeller_Julian:nnn #1#2#3
  {
    \int_mod:nn
      {
          (#3)
        + \int_div_truncate:nn { 26 * ( #2 + \c_one ) } \c_ten
        + (#1)
        + \int_div_truncate:nn {#1} \c_four
        + \c_five
      }
      { \c_seven }
  }
%    \end{macrocode}
% \end{macro}
%
% \begin{macro}{\zhtime}
% 输出时间。
%    \begin{macrocode}
\cs_new_nopar:Npn \zhtime #1
  { \@@_time:ww #1 \q_stop }
\group_begin:
\char_set_lccode:nn { `\; } { `\: }
\tl_to_lowercase:n
  {
    \group_end:
    \cs_new_nopar:Npn \@@_time:ww #1 ; #2 \q_stop
      {
        \zhnum_check_time:Nn \zhnum_integer:n {#1} \c_@@_hour_tl
        \zhnum_check_time:Nn \zhnum_integer:n {#2} \c_@@_minute_tl
      }
  }
%    \end{macrocode}
% \end{macro}
%
% \begin{macro}{\zhcurrtime}
% 输出当前时间。
%    \begin{macrocode}
\cs_new_nopar:Npn \zhcurrtime
  {
    \zhnum_check_time:Nn \zhnum_integer:n
      { \int_div_truncate:nn \tex_time:D { 60 } } \c_@@_hour_tl
    \zhnum_check_time:Nn \zhnum_integer:n
      { \int_mod:nn \tex_time:D { 60 } } \c_@@_minute_tl
  }
%    \end{macrocode}
% \end{macro}
%
% \begin{macro}[internal]{\zhnum_digit_map:n}
% 阿拉伯数字与中文数字的映射。
%    \begin{macrocode}
\cs_new_nopar:Npn \zhnum_digit_map:n #1
  { \tl_use:c { c_@@_ \zhnum_int_to_word:n {#1} _tl } }
%    \end{macrocode}
% \end{macro}
%
% \begin{macro}[internal]{\zhnum_scale_map:n,\zhnum_scale_map_loop:n}
% 大数系统的映射。
%    \begin{macrocode}
\cs_new_nopar:Npn \zhnum_scale_map:n #1
  {
    \cs_if_exist_use:cF { c_@@_scale_ \zhnum_int_to_word:n {#1} _tl }
      { \zhnum_scale_map_hook:n {#1} }
  }
\cs_new_nopar:Npn \zhnum_scale_map_loop:n #1
  { \zhnum_scale_map:n { \int_mod:nn {#1} \g_@@_scale_int } }
\int_new:N \g_@@_scale_int
\int_set_eq:NN \g_@@_scale_int \c_eleven
\cs_new_eq:NN \zhnum_scale_map_hook:n \zhnum_scale_map_loop:n
%    \end{macrocode}
% \end{macro}
%
% \begin{macro}{\zhnumExtendScaleMap}
%    \begin{macrocode}
\NewDocumentCommand \zhnumExtendScaleMap { > { \TrimSpaces } o m }
  {
    \int_zero:N \l_tmpa_int
    \clist_map_inline:nn {#2}
      {
        \int_incr:N \l_tmpa_int
        \tl_set:Nx \l_tmpa_tl
          { c_@@_scale_ \zhnum_int_to_word:n { \l_tmpa_int + \c_eleven } _tl }
        \tl_if_exist:cF { \l_tmpa_tl }
          { \int_incr:N \g_@@_scale_int }
        \tl_set:cn { \l_tmpa_tl } {##1}
      }
    \IfNoValueF {#1}
      { \cs_set:Npn \zhnum_scale_map_hook:n ##1 {#1} }
  }
%    \end{macrocode}
% \end{macro}
%
% \begin{macro}[internal]{\zhnum_int_to_word:n}
% 将整数转换成英文单词。
%    \begin{macrocode}
\cs_new_nopar:Npn \zhnum_int_to_word:n #1
  {
    \if_case:w \etex_numexpr:D #1 \scan_stop:
           zero
      \or: one
      \or: two
      \or: three
      \or: four
      \or: five
      \or: six
      \or: seven
      \or: eight
      \or: nine
      \or: ten
      \or: eleven
    \else:
      \int_case:nnn {#1}
        {
          { 20 } { twenty }  { 30  } { thirty }
          { 40 } { forty }   { 200 } { two_hundred }
        }
        { \int_to_roman:n {#1} }
    \fi:
  }
%    \end{macrocode}
% \end{macro}

% 根据需要设置中文阿拉伯数字。
%    \begin{macrocode}
\keys_define:nn { zhnum / options }
  {
    -   .tl_set:N = \l_@@_minus_tl ,
    -0  .tl_set:N = \l_@@_null_tl ,
    E2  .tl_set:N = \l_@@_hundred_tl ,
    E3  .tl_set:N = \l_@@_thousand_tl ,
    FE2 .tl_set:N = \l_@@_financial_hundred_tl ,
    FE3 .tl_set:N = \l_@@_financial_thousand_tl
  }
\clist_map_inline:nn
  { 0 , 1 , 2 , 3 , 4 , 5 , 6 , 7 , 8 , 9 , 10 , 20 , 30 , 40 , 200 }
  {
    \keys_define:nn { zhnum / options }
      { #1 .tl_set:c = { l_@@_ \zhnum_int_to_word:n {#1} _tl } }
    \int_compare:nNnF {#1} > \c_ten
      {
        \keys_define:nn { zhnum / options }
          { F#1 .tl_set:c = { l_@@_financial_ \zhnum_int_to_word:n {#1} _tl } }
      }
  }
\clist_map_inline:nn
  { 4 , 8 , 12 , 16 , 20 , 24 , 28 , 32 , 36 , 40 , 44 }
  {
    \keys_define:nn { zhnum / options }
      { E#1 .tl_set:c = { l_@@_scale_ \zhnum_int_to_word:n { #1 / 4 } _tl } }
  }
\clist_map_inline:nn
  {
    dot , and , parts , year , month , day , weekday , hour , minute
    mon , tue , wed , thu , fri , sat , sun
  }
  { \keys_define:nn { zhnum / options } { #1 .tl_set:c = { l_@@_ #1 _tl } } }
%    \end{macrocode}
%
% \begin{macro}[internal]
%   {
%     \zhnum_set_digits_map:nn,
%     \zhnum_set_digits_map:nnn,
%     \zhnum_set_financial_map:nn,
%     \zhnum_set_financial_map:nnn,
%     \l_@@_cfg_map_prop,
%     \l_@@_cfg_map_var_prop,
%     \l_@@_cfg_map_finan_prop
%   }
% 将配置文件中的中文数字保存到 \texttt{prop} 变量中。
%    \begin{macrocode}
\cs_new_protected_nopar:Npn \zhnum_set_digits_map:nn #1#2
  { \prop_put:Nnn \l_@@_cfg_map_prop {#1} {#2} }
\cs_new_protected_nopar:Npn \zhnum_set_digits_map:nnn #1#2#3
  {
    \prop_put_if_new:Nnn \l_@@_cfg_map_prop {#1} {#3}
    \prop_put:Nnn \l_@@_cfg_map_var_prop {#1_#2} {#3}
  }
\cs_new_protected_nopar:Npn \zhnum_set_financial_map:nn #1#2
  { \prop_put:Nnn \l_@@_cfg_map_finan_prop {#1} {#2} }
\cs_new_protected_nopar:Npn \zhnum_set_financial_map:nnn #1#2#3
  {
    \prop_put_if_new:Nnn \l_@@_cfg_map_finan_prop {#1} {#3}
    \prop_put:Nnn \l_@@_cfg_map_var_prop { financial_#1_#2 } {#3}
  }
\prop_new:N \l_@@_cfg_map_prop
\prop_new:N \l_@@_cfg_map_var_prop
\prop_new:N \l_@@_cfg_map_finan_prop
%    \end{macrocode}
% \end{macro}
%
% \begin{macro}[internal]
%   {
%     \zhnum_reset_config:,
%     \zhnum_check_simp:nn,
%     \zhnum_check_financial:nn,
%     \zhnum_set_zero:,
%     \zhnum_set_week_day:
%   }
% 将 \texttt{prop} 表转化到单独的 \texttt{tl} 变量。
%    \begin{macrocode}
\cs_new_protected_nopar:Npn \zhnum_reset_config:
  {
    \prop_map_function:NN \l_@@_cfg_map_prop \zhnum_check_simp:nn
    \bool_if:NF \l_@@_reset_bool
      { \prop_map_function:NN \l_@@_cfg_map_prop \zhnum_check_financial:nn }
    \zhnum_set_zero:
    \zhnum_set_week_day:
  }
\cs_new_protected_nopar:Npn \zhnum_check_simp:nn #1#2
  {
    \@@_check_simp_aux:nn {#1} {#2}
    \prop_get:NnNT \l_@@_cfg_map_finan_prop {#1} \l_tmpa_tl
      { \exp_args:Nno \@@_check_simp_aux:nn { financial_ #1 } \l_tmpa_tl }
  }
\cs_new_protected_nopar:Npn \@@_check_simp_aux:nn #1#2
  {
    \prop_get:NnNTF \l_@@_cfg_map_var_prop { #1 _trad } \l_tmpa_tl
      {
        \prop_get:NnNTF \l_@@_cfg_map_var_prop { #1 _simp } \l_tmpb_tl
          {
            \tl_set:cx { l_@@_ #1 _tl }
              {
                \exp_not:n { \bool_if:NTF \l_@@_simp_bool }
                  { \exp_not:o \l_tmpb_tl } { \exp_not:o \l_tmpa_tl }
              }
          }
          {
            \tl_set:cx { l_@@_ #1 _tl }
              {
                \exp_not:n { \bool_if:NTF \l_@@_simp_bool }
                  { \exp_not:n {#2} } { \exp_not:o \l_tmpa_tl }
              }
          }
      }
      { \tl_set:cn { l_@@_ #1 _tl } {#2} }
  }
\cs_new_protected_nopar:Npn \zhnum_check_financial:nn #1#2
  {
    \prop_get:NnNTF \l_@@_cfg_map_finan_prop {#1} \l_tmpa_tl
      {
        \zhnum_assgin_const_tl:nn { c_@@_ #1 _tl }
          {
            \exp_not:n { \bool_if:NTF \l_@@_normal_bool }
              { \exp_not:c { l_@@_ #1 _tl } }
              { \exp_not:c { l_@@_financial_ #1 _tl } }
          }
      }
      {
        \zhnum_assgin_const_tl:nn
          { c_@@_ #1 _tl } { \exp_not:c { l_@@_ #1 _tl } }
      }
  }
\cs_new_protected_nopar:Npn \zhnum_set_zero:
  {
    \tl_set:Nx \l_@@_zero_tl
      {
        \exp_not:n { \bool_if:nTF \l_@@_null_bool }
          { \exp_not:o \l_@@_null_tl } { \exp_not:o \l_@@_zero_tl }
      }
  }
\cs_new_protected_nopar:Npn \zhnum_set_week_day:
  {
    \tl_set:Nx \l_@@_mon_tl
      { \exp_not:N \c_@@_weekday_tl \exp_not:o \l_@@_one_tl }
    \tl_set:Nx \l_@@_tue_tl
      { \exp_not:N \c_@@_weekday_tl \exp_not:o \l_@@_two_tl }
    \tl_set:Nx \l_@@_wed_tl
      { \exp_not:N \c_@@_weekday_tl \exp_not:o \l_@@_three_tl }
    \tl_set:Nx \l_@@_thu_tl
      { \exp_not:N \c_@@_weekday_tl \exp_not:o \l_@@_four_tl }
    \tl_set:Nx \l_@@_fri_tl
      { \exp_not:N \c_@@_weekday_tl \exp_not:o \l_@@_five_tl }
    \tl_set:Nx \l_@@_sat_tl
      { \exp_not:N \c_@@_weekday_tl \exp_not:o \l_@@_six_tl }
    \tl_set:Nx \l_@@_sun_tl
      { \exp_not:N \c_@@_weekday_tl \exp_not:o \l_@@_day_tl }
    \bool_if:NF \l_@@_reset_bool
      {
        \clist_map_inline:nn { mon , tue , wed , thu , fri , sat , sun }
          {
            \zhnum_assgin_const_tl:nn
              { c_@@_ ##1 _tl } { \exp_not:c { l_@@_ ##1 _tl } }
          }
        \bool_set_true:N \l_@@_reset_bool
      }
  }
\cs_new_eq:NN \zhnum_assgin_const_tl:nn \tl_const:cx
\AtEndOfPackage
  { \cs_set_eq:NN \zhnum_assgin_const_tl:nn \tl_set:cx }
%    \end{macrocode}
% \end{macro}
%
% \begin{macro}[internal]{\zhnum_load_cfg:n}
% 根据选定编码载入配置文件。
%    \begin{macrocode}
\cs_new_protected_nopar:Npn \zhnum_load_cfg:n #1
  {
    \str_if_eq_x:nnTF {#1} { \l_@@_last_encoding_tl }
      { \zhnum_reset_config: }
      {
        \file_if_exist_input:nTF { zhnumber - #1 .cfg }
          {
            \bool_set_false:N \l_@@_reset_bool
            \zhnum_set_byte_range:n {#1}
            \tl_set:Nx \l_@@_last_encoding_tl {#1}
            \prop_clear:N \l_@@_cfg_map_prop
            \prop_clear:N \l_@@_cfg_map_var_prop
            \prop_clear:N \l_@@_cfg_map_finan_prop
            \zhnum_set_catcode:
          }
          {
            \msg_error:nnx { zhnumber } { file-not-found } {#1}
            \use_none:nn
          }
        \zhnum_reset_config:
        \zhnum_reset_catcode:
      }
  }
\tl_new:N \l_@@_last_encoding_tl
\bool_new:N \l_@@_reset_bool
\msg_new:nnnn  { zhnumber } { file-not-found }
  { File~`#1'~not~found. }
  {
    The~requested~file~could~not~be~found~in~the~current~directory,~
    in~the~TeX~search~path~or~in~the~LaTeX~search~path.
  }
%    \end{macrocode}
% \end{macro}
%
% \begin{macro}[internal]{\zhnum_set_catcode:}
% 设置与恢复配置文件前后的 catcode。\pdfTeX{} 需要将汉字的首字节设置为活动字符。
%    \begin{macrocode}
\cs_new_protected_nopar:Npn \zhnum_set_catcode:
  {
    \bool_if:NTF \l__kernel_expl_bool
      { \cs_set_eq:NN \zhnum_reset_catcode: \zhnum_reset_active: }
      {
        \cs_set_protected_nopar:Npn \zhnum_reset_catcode:
          { \ExplSyntaxOff \zhnum_reset_active: }
      }
    \zhnum_set_active:
  }
\cs_new_protected_nopar:Npn \zhnum_reset_catcode:
  { \ExplSyntaxOff \zhnum_reset_active: }
\if_predicate:w \pdftex_if_engine_p:
  \cs_new_protected_nopar:Npn \zhnum_set_byte_range:n #1
    {
      \str_case_x:nnTF {#1}
        {
          { gbk }  { \int_set:Nn \l_@@_byte_min_int { "81 } }
          { big5 } { \int_set:Nn \l_@@_byte_min_int { "A1 } }
        }
        { \int_set:Nn \l_@@_byte_max_int { "FE } }
        {
          \int_set:Nn \l_@@_byte_min_int { "E0 }
          \int_set:Nn \l_@@_byte_max_int { "EF }
        }
    }
  \int_new:N \l_@@_byte_min_int
  \int_new:N \l_@@_byte_max_int
  \cs_new_protected_nopar:Npn \zhnum_set_active:
    {
      \tl_clear:N \l_@@_reset_catcode_tl
      \int_step_function:nnnN
        { \l_@@_byte_min_int } { \c_one }
        { \l_@@_byte_max_int } \@@_set_active:n
    }
  \cs_new_protected_nopar:Npn \@@_set_active:n #1
    {
      \int_compare:nNnF { \char_value_catcode:n {#1} } = \c_thirteen
        {
          \tl_put_right:Nx \l_@@_reset_catcode_tl
            { \char_set_catcode:nn {#1} { \char_value_catcode:n {#1} } }
          \char_set_catcode_active:n {#1}
        }
    }
  \cs_new_protected_nopar:Npn \zhnum_reset_active:
    { \tl_use:N \l_@@_reset_catcode_tl }
  \tl_new:N \l_@@_reset_catcode_tl
\else:
  \cs_new_eq:NN \zhnum_set_byte_range:n \use_none:n
  \cs_new_eq:NN \zhnum_set_active:   \prg_do_nothing:
  \cs_new_eq:NN \zhnum_reset_active: \prg_do_nothing:
\fi:
%    \end{macrocode}
% \end{macro}
%
% \begin{macro}{encoding,style,null,reset}
% 宏包设置选项。
%    \begin{macrocode}
\keys_define:nn { zhnum / options }
  {
    encoding         .choices:nn =
      { UTF8 , GBK , Big5 }
      {
        \tl_set:Nx \l_@@_encoding_tl
          { \exp_args:No \tl_expandable_lowercase:n { \l_keys_choice_tl } }
        \zhnum_load_cfg:n { \l_@@_encoding_tl }
      } ,
    encoding          .default:n = { GBK } ,
    encoding / Bg5       .meta:n = { encoding = Big5 } ,
    encoding / unknown   .code:n =
      { \msg_error:nnn { zhnumber } { encoding-invalid } {#1} } ,
    style .multichoice: ,
    style / Normal       .code:n =
      {
        \bool_set_false:N \l_@@_ancient_bool
        \bool_set_true:N  \l_@@_normal_bool
      } ,
    style / Financial    .code:n =
      {
        \bool_set_false:N \l_@@_ancient_bool
        \bool_set_false:N \l_@@_normal_bool
      } ,
    style / Ancient      .code:n =
      {
        \bool_set_true:N \l_@@_ancient_bool
        \bool_set_true:N \l_@@_normal_bool
      } ,
    style / Simplified   .code:n = { \bool_set_true:N  \l_@@_simp_bool } ,
    style / Traditional  .code:n = { \bool_set_false:N \l_@@_simp_bool } ,
    style             .default:n = { Normal , Simplified } ,
    null             .bool_set:N = \l_@@_null_bool ,
    time .choice: ,
    time / Chinese       .code:n = { \bool_set_true:N \l_@@_time_bool } ,
    time / Arabic        .code:n = { \bool_set_false:N  \l_@@_time_bool } ,
    time              .default:n = { Arabic } ,
    reset                .code:n = { \zhnum_reset_config: } ,
  }
\tl_new:N \l_@@_encoding_tl
\msg_new:nnnn { zhnumber } { encoding-invalid }
  { The~encoding~`#1'~is~invalid. }
  {
    Available~encoding~are~`UTF8',~`GBK'~and~`Big5'.
  }
%    \end{macrocode}
% \end{macro}
%
% \begin{macro}{\zhnumsetup}
% 在文档中设置 \pkg{zhnumber} 的接口。
%    \begin{macrocode}
\NewDocumentCommand \zhnumsetup { m }
  {
    \keys_set:nn { zhnum / options } {#1}
    \tex_ignorespaces:D
  }
%    \end{macrocode}
% \end{macro}
%
% 初始化设置和执行宏包选项。
%    \begin{macrocode}
\keys_set:nn { zhnum / options } { style , time }
\ProcessKeysOptions { zhnum / options }
%    \end{macrocode}
%
% 如果没有选定编码,则根据引擎自动设置编码。
%    \begin{macrocode}
\tl_if_empty:NT \l_@@_encoding_tl
  {
    \pdftex_if_engine:TF
      { \keys_set:nn { zhnum / options } { encoding = GBK } }
      { \keys_set:nn { zhnum / options } { encoding = UTF8 } }
  }
%    \end{macrocode}
%
%    \begin{macrocode}
%</package>
%    \end{macrocode}
%
% \section{中文数字配置文件}
% \label{sec:zhnum-map}
%
%    \begin{macrocode}
%<*config>
%    \end{macrocode}
%
%    \begin{macrocode}
%<*!big5>
\zhnum_set_digits_map:nnn { minus } { simp }  { 负 }
\zhnum_set_digits_map:nnn { minus } { trad }  { 負 }
%</!big5>
%<*big5>
\zhnum_set_digits_map:nn { minus }       { 負 }
%</big5>
\zhnum_set_digits_map:nn { zero }        { 零 }
%<*!big5>
\zhnum_set_digits_map:nn { null }        { 〇 }
%</!big5>
%<*big5>
\zhnum_set_digits_map:nn { null }        { ○ }
%</big5>
\zhnum_set_digits_map:nn { one }         { 一 }
\zhnum_set_digits_map:nn { two }         { 二 }
\zhnum_set_digits_map:nn { three }       { 三 }
\zhnum_set_digits_map:nn { four }        { 四 }
\zhnum_set_digits_map:nn { five }        { 五 }
\zhnum_set_digits_map:nn { six }         { 六 }
\zhnum_set_digits_map:nn { seven }       { 七 }
\zhnum_set_digits_map:nn { eight }       { 八 }
\zhnum_set_digits_map:nn { nine }        { 九 }
\zhnum_set_digits_map:nn { ten }         { 十 }
\zhnum_set_digits_map:nn { hundred }     { 百 }
\zhnum_set_digits_map:nn { thousand }    { 千 }
\zhnum_set_digits_map:nn { twenty }      { 廿 }
\zhnum_set_digits_map:nn { thirty }      { 卅 }
\zhnum_set_digits_map:nn { forty }       { 卌 }
\zhnum_set_digits_map:nn { two_hundred } { 皕 }
%<*!big5>
\zhnum_set_digits_map:nnn { dot } { simp } { 点 }
\zhnum_set_digits_map:nnn { dot } { trad } { 點 }
%</!big5>
%<*big5>
\zhnum_set_digits_map:nn { dot }   { 點 }
%</big5>
\zhnum_set_digits_map:nn { and }   { 又 }
\zhnum_set_digits_map:nn { parts } { 分之 }
\zhnum_set_digits_map:nn { scale_zero }   { }
%<*!big5>
\zhnum_set_digits_map:nnn { scale_one } { simp } { 万 }
\zhnum_set_digits_map:nnn { scale_one } { trad } { 萬 }
\zhnum_set_digits_map:nnn { scale_two } { simp } { 亿 }
\zhnum_set_digits_map:nnn { scale_two } { trad } { 億 }
%</!big5>
%<*big5>
\zhnum_set_digits_map:nn { scale_one }    { 萬 }
\zhnum_set_digits_map:nn { scale_two }    { 億 }
%</big5>
\zhnum_set_digits_map:nn { scale_three }  { 兆 }
\zhnum_set_digits_map:nn { scale_four }   { 京 }
\zhnum_set_digits_map:nn { scale_five }   { 垓 }
\zhnum_set_digits_map:nn { scale_six }    { 秭 }
\zhnum_set_digits_map:nn { scale_seven }  { 穰 }
%<*!big5>
\zhnum_set_digits_map:nnn { scale_eight } { simp }  { 沟 }
\zhnum_set_digits_map:nnn { scale_eight } { trad }  { 溝 }
\zhnum_set_digits_map:nnn { scale_nine  } { simp }  { 涧 }
\zhnum_set_digits_map:nnn { scale_nine  } { trad }  { 澗 }
%</!big5>
%<*big5>
\zhnum_set_digits_map:nn { scale_eight }  { 澗 }
%</big5>
\zhnum_set_digits_map:nn { scale_ten }    { 正 }
%<*!big5>
\zhnum_set_digits_map:nnn { scale_eleven } { simp } { 载 }
\zhnum_set_digits_map:nnn { scale_eleven } { trad } { 載 }
%</!big5>
%<*big5>
\zhnum_set_digits_map:nn { scale_eleven } { 載 }
%</big5>
\zhnum_set_digits_map:nn { year }    { 年 }
\zhnum_set_digits_map:nn { month }   { 月 }
\zhnum_set_digits_map:nn { day }     { 日 }
%<*!big5>
\zhnum_set_digits_map:nnn { hour } { simp } { 时 }
\zhnum_set_digits_map:nnn { hour } { trad } { 時 }
%</!big5>
%<*big5>
\zhnum_set_digits_map:nn { hour }    { 時 }
%</big5>
\zhnum_set_digits_map:nn { minute }  { 分 }
\zhnum_set_digits_map:nn { weekday } { 星期 }
\zhnum_set_financial_map:nn { null }     { 零 }
\zhnum_set_financial_map:nn { zero }     { 零 }
\zhnum_set_financial_map:nn { one }      { 壹 }
\zhnum_set_financial_map:nn { two }      { 貳 }
%<*!big5>
\zhnum_set_financial_map:nnn { three } { simp } { 叁 }
\zhnum_set_financial_map:nnn { three } { trad } { 叄 }
%</!big5>
%<*big5>
\zhnum_set_financial_map:nn { three }    { 參 }
%</big5>
\zhnum_set_financial_map:nn { four }     { 肆 }
\zhnum_set_financial_map:nn { five }     { 伍 }
%<*!big5>
\zhnum_set_financial_map:nnn { six }   { simp } { 陆 }
\zhnum_set_financial_map:nnn { six }   { trad } { 陸 }
%</!big5>
%<*big5>
\zhnum_set_financial_map:nn { six }      { 陸 }
%</big5>
\zhnum_set_financial_map:nn { seven }    { 柒 }
\zhnum_set_financial_map:nn { eight }    { 捌 }
\zhnum_set_financial_map:nn { nine }     { 玖 }
\zhnum_set_financial_map:nn { ten }      { 拾 }
\zhnum_set_financial_map:nn { hundred }  { 佰 }
\zhnum_set_financial_map:nn { thousand } { 仟 }
%    \end{macrocode}
%
%    \begin{macrocode}
%</config>
%    \end{macrocode}
%
% \end{implementation}
%
% \Finale
%
\endinput

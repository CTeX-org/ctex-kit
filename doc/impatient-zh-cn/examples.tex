% This is part of the book TeX for the Impatient.
% Copyright (C) 2003 Paul W. Abrahams, Kathryn A. Hargreaves, Karl Berry.
% See file fdl.tex for copying conditions.

\input macros
\chapter{例子}

\chapterdef{examples}


這一章中包括了一組例子, 來幫助你熟悉 \TeX, 同時這些例子還展示了如何使用 \TeX\ 來完成各種排版工作.
每個例子都有一個放在左頁的 \TeX\ 排版的結果和放在右頁的相對應的 \TeX\ 輸入文本.
你可以把這些例子作為模仿的樣式, 也可以用來找到你想要的效果的實現命令.
不過要注意的是, 這些例子僅能展示 \TeX\ $900$ 條左右的命令的一小部分.

這裡的某些例子是其義自現的---也就是說, 它們在介紹每個所排印出來的功能.
這個介紹很粗略, 因為沒有足夠的篇幅來講術所有你想得到的信息.
一個命令的速查摘要 (\chapterref{capsule}) 和索引可以幫你來找到例子中的每個 \TeX\ 功能.

因為我們在設計這些例子時, 把很多的東西放在一起描述,
因此, 這些例子展示了很多的排版效果.
這些例子一般並\emph{不}是好的排版實踐模版.
比如例~8 把有此公式編號放在左邊, 又把另一些放在了右邊.
你永遠不會在一個實際的科學出版物中使用這樣的公式編號.

\xrdef{xmphead}
除了第一個例子以外, 每個例子都由一個叫 |\xmpheader| 的宏開始 (見 \xref{macro}).
我們這樣做是為了節省輸入文本的篇幅, 
否則每個例子開頭你都會看到幾行你先前已經看到的內容.
|\xmpheader| 會排印出例子的標題和標題後的空白.
你可以參見沒有使用 |\xmpheader| 的第一個例子是如何實現這一點的,
然後你就能模仿它了.
除了 |\xmpheader|, 在這裡使用的每個命令都是在 \plainTeX\ 中定義過的.

% The first example does the necessary eject here.
{%
   \let\bye = \relax % We don't want to obey \bye in the example input.
   % These switches can't be done by a macro since \bye is outer.
   \doexamples {xmptext}% Typeset the actual examples.
}%


\endchapter
\byebye
